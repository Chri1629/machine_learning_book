\documentclass[12pt, a4paper,titlepage,openany]{article}
\usepackage[italian]{babel}
\usepackage[T1]{fontenc}
\usepackage[table]{xcolor}
\definecolor{ao(english)}{rgb}{0.0, 0.5, 0.0}
\usepackage{float}
\restylefloat{table,figure}
\usepackage{graphicx}	
\usepackage[utf8]{inputenc}
\usepackage{amsmath}
\usepackage{fancyhdr}
\pagestyle{fancy}
\usepackage{amssymb}
\usepackage{geometry}
\usepackage{url}
\usepackage{hyperref}

\geometry{a4paper,top=2cm,bottom=2cm,left=3cm,right=3cm,heightrounded,bindingoffset=5mm}
\usepackage{amsthm}
\usepackage{enumerate}
\usepackage{verbatim}
\usepackage{setspace}
\usepackage{soul}
\title{Dispensa di Machine Learning}
\author{Federico Luzzi \\ Christian Uccheddu}
\date{}
\usepackage{enumitem}

\theoremstyle{plain}
\newtheorem{thm}{Teorema}[section]
\newtheorem{cor}[thm]{Corollario}
\newtheorem{lem}[thm]{Lemma}
\newtheorem{prop}[thm]{Proposizione}
\theoremstyle{definition}
\newtheorem{defn}{Definizione}[section]

\theoremstyle{remark}
\newtheorem{oss}{Osservazione}
\begin{document}
		\begin{titlepage}
		
		\newcommand{\HRule}{\rule{\linewidth}{0.5mm}} 
		
		\centering 
		

		{\setstretch{1.1} 
			\textsc{\LARGE università degli studi di milano-bicocca}\\[0.5cm] 
		}
		\includegraphics[width = .5\textwidth]{Logo-Scuola.png}\\[1cm] 
		\textsc{\Large Scuola di Scienze }\\[0.25cm] 
		\textsc{\large Corso di laurea magistrale in Data Science}\\[0.75cm] 

		
		\HRule \\[0.4cm]
		{\setstretch{1.5} 
			{\huge \bfseries Dispensa di Machine Learning }\\[0.4cm] % Title of your document
		}
		\HRule \\[1.5cm]
		

		\begin{center}
			\large\textbf{Autori:} \\ Federico Luzzi \\ Christian Uccheddu
		\end{center}
		\vspace{3.5cm}
		{
			{\large \textsc{anno accademico 2019/2020}}\\[1cm] 
		}
		
		\vfill 
	\end{titlepage}
	\newpage
	\textit{Questa vuole essere una dispensa riassuntiva del corso Machine Learning del corso di laurea magistrale in Data Science dell'Università degli Studi di Milano Bicocca. La scrittura di questa dispensa è stata possibile solo grazie a tutto il materiale messo a disposizione dal Professore Fabio Stella.
	Il repository di questa dispensa è disponibile a questo link:  \href{https://github.com/Chri1629/machine_learning_book.git}{\url{https://github.com/Chri1629/machine_learning_book.git}}
	\\ Ricorda "Il sapere si accresce solo se condiviso".}

	
	
	
	\newpage
\pagenumbering{Roman}
%\maketitle
\tableofcontents
\clearpage
\pagenumbering{arabic}
\restylefloat{table,figure}
\pagestyle{fancy}
\cfoot{\thepage}
\renewcommand{\footrulewidth}{0.25pt}


\section{Dati}
Gli ambiti più importanti nei quali vengono applicate tecniche di machine learning nella vita reale sono diversi, in particolare si usano in:
finanza, sanità, agricoltura, e-commerce, social, chatbot, sensoristica (come i veicoli a guida autonoma).

Vista la grande mole di dati la necessità è capire come trattare i dati.

\textbf{L'obiettivo del Machine Learning è sviluppare metodologia per dare valore ai dati in funzione di una particolare domanda che ci stiamo facendo.}

Tipicamente le tecninche di machine learning si dividono nelle seguenti tre macro categorie:
\begin{itemize}
	\item Apprendimento supervisionato o predittivo: qualcuno ha gia catalogato ad esempio delle immagini o dei dati e noi prendendo questi modelli dovremmo essere in grado di predire.
	\item Apprendimento descrittivo: ci sono delle funzioni obiettivo che vanno ottimizzate. Non usiamo etichette della singola istanza ma in qualche modo sappiamo dove arrivare.
	\item Apprendimento rinforzato: E' quello più utile in questa epoca: funziona sui premi.
\end{itemize}

In questo corso ci concentreremo sui primi due tipi di apprendimento.

L' \textit{apprendimento supervisionato} si divide a sua volta nelle seguenti due categorie:
\begin{itemize}
	\item Classificazione: quando il problema consiste nel dividere in classi delle quantità discrete.
	\item Regressione: quando il problema consiste nel ricostruire una certa variabile date delle condizioni pregresse.
\end{itemize}

L' \textit{apprendimento non supervisionato} si divide a sua volta in:
\begin{itemize}
	\item Clustering: Quando il problema consistere nel ricostruire delle classi delle istanze che ci vengono consegnate, senza sapere nulla sulla storia pregressa.
	\item Associativa: Quando il problema consiste nel scoprire pattern che descrivono bene caratteristiche associate ad un certo fenomeno.
\end{itemize}

Per alcuni compiti la correlazione statistica va benissimo, in alcuni casi però essa è addirittura deleteria.  Se infatti provassi a vedere la correlazione tra il numero di omicidi in america e il numero di fondi investiti sulla ricerca scientifica vedrei che statisticamente sono strettamente correlate. Questo è un no-sense ed è il classico esempio di \textit{correlazione spuria.}

A confermare questa visione in cui il modello per una certa trattazione dati è fondamentale è il cosidetto paradosso di Simpson.

\textbf{Paradosso di Simpson:} Se uso solo i dati senza modello no c'è alcun modo di scoprire la verità.


\subsection{Data types}

Il primo passo fondamentale è sicuramente quello di prendere confidenza coi dati. E' fondamentale capire la natura intrinseca dei dati che abbiamo a disposizione, di solito essi sono organizzati in strutture che chiamiamo \textbf{dataset}. Per le analisi con le tecniche di Machine Learning esse non sono altro che tabelle fatte da righe e da colonne.

Le colonne del dataset sono chiamate \textbf{Attributi.}

Le righedel dataset sono chiamate \textbf{Istanze.}

Ogni attributo è caratterizzato dal fatto di avere un tipo, la sua conoscenza è fondamentale perché ci permette di sapere le proprietà che questo possiede.
In particolare gli attributi si dividono in due grandi gruppi:
\begin{itemize}
	\item Categorici:
	\begin{itemize}
		\item Nominali: ad esempio il colore degli occhi.
		\item Ordinali: ad esempio possono essere i giudizi.
	\end{itemize}
	\item Numerici:
	\begin{itemize}
		\item Intervallo: ammettono operazioni di somma e sottrazione. 
		\item Ratio: possiamo applicare tutte le operazioni logico/matematiche.
	\end{itemize}
\end{itemize}


\begin{figure}[H]
	\centering
	\includegraphics[height=0.5 \linewidth]{introduction/pict/attributi.png}
	\caption{Distinzione degli attributi con le relative proprietà}
\end{figure}
Dall'alto al basso il livello gerarchico sale e le proprietà aumentano.

C'è un'ulteriore divisione che può essere fatta ed è quella in:
Si possono anche dividere in attributi \textit{discreti} che possono essere:
\begin{itemize}
	\item Discreti (in cui la serie dei valori è finita o una infinità numerabile), che a loro volta si suddividono in:
	\begin{itemize}
		\item Categorici
		\item Numerici
		\item Binari: sono i più particolari da trattare, e hanno una serie di proprietà strane.
	\end{itemize}
	\item Continui, i cui valori sono numeri reali.
\end{itemize}

\subsection{Data exploration}

Dobbiamo però anche sapere come esplorare i dati in modo intelligente usando tutti i nostri strumenti a nostra disposizione. Diamo ora un elenco degli strumenti statistici più utili che ci permetton di avere un'analisi completa dei dati.

\subsubsection{Definizioni}
\begin{defn}
	Si definisce \textbf{quantile} $\alpha$ il valore $q_{\alpha}$ che divide la popolazione in due parti proporzionali rispettivamente ad $\alpha$ e a $(1 - \alpha)$ e caratterizzate da valori rispettivamente minori e maggiori di $q_{\alpha}$
\end{defn}

Un quantile molto importante è il quantile di ordine $\frac{1}{2}$ e si chiama \textbf{mediana} che è quello che ha esattamente minori di lui  metà dei dati.

\begin{defn}
	Si definisce \textbf{media} il seguente valore:
	\[mean = \frac{1}{n}\sum_{i=1}^{n}x_i\]
\end{defn}


La media non è un buon modo di visualizzare i dati perché dice poco riguardo alla loro distribuzione. Siccome la media è basata su singole osservazioni è fortemente soggetta a variazioni quando si hanno valori fortemente discostati dalla distribuzione dei dati.
Tali valori si chiamano \textit{outlier} e sono particolarmente interessanti da trattare.

Per ovviare a questo problema si è soliti usare  la \textbf{media trimmed} in cui si buttano via il valore più piccolo e il valore più grande. 
\begin{defn}
	Si definisce \textbf{media trimmed} il seguente valore:
	\[mean_{trimmed} = \frac{1}{n}\sum_{i=1}^{n}x_i  \qquad x_{i} = x - \max{x}- \min{x}\]
\end{defn}

Se si trova un grosso scostamento tra la media e la media trimmed probabilmente si ha la presenza di almeno un outlier.

\begin{defn}
	Si definisce \textbf{range} il seguente valore:
	\[ range = \max{x}- \min{x}\]
\end{defn}

Esso serve a quantificare la dispersione dei dati, ovviamente questo valore può essere fuorviante qualora i valori siano concentrati in una stretta banda di valori. Per ovviare a questo problema si usa la varianza.

\begin{defn}
	Si definisce \textbf{varianza} il seguente valore:
	\[ var =  \sigma^{2} =\frac{1}{n - 1 }\sum_{i = 1}^{n} (x_{i} - \bar{x})^{2}\]	
\end{defn}

E' preferibile però usare la deviazione standard in quanto per come è definita risulta essere della stessa unità di misura dei dati.
\begin{defn}
	Si definisce \textbf{deviazione standard} il seguente valore:
	\[ std = \sigma =  \sqrt{\sigma^{2}} \]
\end{defn}

Essendo queste due grandezze definite a partire dalla media esse soffrono dello stesso problema: sono fortemente condizionate dagli outlier, con lo stesso metodo precedente si definiscono quindi altre due grandezze in grado di ovviare a questo problema.

\begin{defn}
	Si definisce \textbf{deviazione media assoluta} il seguente valore:
	\[ AAD =  \frac{1}{n}\sum_{i=1}^{n}|x_i- mean|\]
\end{defn}

\begin{defn}
	Si definisce \textbf{deviazione mediana assoluta} il seguente valore:
	\[ MAD = mediana(x_{1}-mean, x_{2}-mean, ..., x_{n0}-mean )\]
\end{defn}

E' utile anche definire il range interquartile (IQR) sempre per ovviare alla presenza di outliar.

\begin{defn}
	Si definisce \textbf{deviazione mediana assoluta} il seguente valore:
	\[ IQR = q_{75\%} - q_{25\%}\]
\end{defn}

Ci sono anche diverse grandezze utili da definire nei casi in cui si ha a che fare con copppie di attributi, in tal caso:

\begin{defn}
	Si definisce \textbf{covarianza} il seguente valore:
\[cov(X,Y) = \frac{1}{m -1}\sum_{i = 1}^{m} (x_{i} - \bar{x})(y_{i} - \bar{y})\]
\end{defn}

Per come è costruita questa matrice è necessariamente quadrata, e il suo valore $ij ^{th}$ rappresenta la covarianza tra il valore $i^{th}$ dell'attributo x e il valore $j^{th}$ dell'attributo y.

Un'ulteriore misura dell'associazione tra le coppie di attributi quantitativi che non dipende dalla varianza di ciascun attributo è la seguente:

\begin{defn}
	Si definisce \textbf{correlazione di Pearson} il seguente valore:
	\[ corr(x,y) = \frac{cov(x,y)}{\sqrt{var(x)var(y)}}\]
\end{defn} 
Per come è definita si ha ovviamente che: $cov(x,y) \in [-1,1]$.

E' molto utile visualizzare i dati quando bisogna lavorarci, questo fondamentalmente per due motivi:
\begin{itemize}
	\item Ci permette di trovare pattern tra le variabili che valori puntuali non ci permetterebbero di trovare.
	\item Ci permette di visualizzare i risultati di una lavorazione fatta sui dati.
\end{itemize}

Di seguito proponiamo un rapido elenco dei grafici più utili e descriviamone le varie caratteristiche.

\subsubsection{Visualization}

Possono organizzare i dati in istogrammi caratterizzati dalla presenza di bin: essi indicano la larghezza in cui i dati sono organizzati. A seconda dell'ampiezza che uso posso ottenere due disegni molto diversi come mostrato in figura.


\begin{figure}[H]
	\centering
	\includegraphics[height=0.35 \linewidth]{introduction/pict/istogramma_quant.png}
	\caption{Differenza tra due istogrammi fatti sulla stessa distribuzione di dati ma con numero di bin diversi.}
\end{figure}


Si possono allo stesso modo creare istogrammi per dati qualitativi. La differenza rispetto agli istogrammi sui dati quantitativi è quella per cui ogni bin corrisponde ad una categoria diversa.

\begin{figure}[H]
	\centering
	\includegraphics[height=0.3 \linewidth]{introduction/pict/istogramma_qual.png}
	\caption{Istogramma di dati qualitativi.}
\end{figure}

In questo caso la linea blu disegnata sopra è una linea cumulativa, ossia rappresenta dove si trova il livello sommando tutti i dati che man mano incontro, per come è costruita ovviamente dovrà finire sul valore $100\%$


Un altro modo di rappresentare i dati particolarmente utile è il \textbf{grafico Box and Whiskers} applicato solo ad attributi quantitativi. Rispetto agli istogrammi questo è decisamente più utilizzato perché permette di estrarre decisamente più informazioni. In particolare proponiamo un grafico in cui sono esplicitate le informazioni ottenibili:

\begin{figure}[H]
	\centering
	\includegraphics[height=0.4 \linewidth]{introduction/pict/box_plot.png}
	\caption{Informazioni ottenibili da un box plot.}
\end{figure}


\subsection{Missing replacement}

E' il problema che si occupa di studiare come sostituire i valori mancanti di certi attributi  in un dataset. Essendo un problema di così larga portata possiamo capire bene che una trattazione esaustiva riempirebbe le ore di un intero corso, tuttavia ne forniamo le basi per essere in grado di poter effettuare una analisi dati efficace. 

Le motivazioni principali del perché un database presenta delle mancanze sono le seguenti:
\begin{itemize}
	\item Tale valore non è stato possibile misurarlo.
	\item Non si conosce con esattezza tale valore.
	\item Si è verificato qualche errore durante la presa dati.
	\item Quel determinato attributo fino ad un certo punto dell'analisi non è mai stato considerati importante e quindi non è mai stato registrato.
\end{itemize}

Ci sono diversi metodi che ci permettono di riempire quel valore mancante, di seguito diamo una rapida trattazione dei più usati:

\begin{itemize}
	\item \textbf{Record removal:} è il metodo più drastico e consistere nell'eliminare tutta l'istanza che contiene quel valore. Non viene usato frequentemente perché si rischia di perdere valori che possono essere fondamentali per l'analisi dati.
	
	\item \textbf{Global constant:} consiste nel sostituire tutti i missing values con un valore costante chiamato \textit{place holder.} Tale metodo non è molto efficiente perché un valore costante non può essere rappresentativo di una distribuzione.
	
	\item \textbf{Manual imputation:} consiste nel sostituire manualmente i missing values tramite delle osservazioni. Lo svantaggio è che è tremendamente svantaggioso dal punto di vista computazionale.
	\item  \textbf{Moda replacement:} consiste nel rimpiazzare tutti i missing values con la moda di quel dato attributo, valgono le stesse considerazioni fatte per il metodo della global constant. Qualora gli attributi fossero continui si effettuerebbe la stessa cosa ma sostituendo la media.
	\item \textbf{Conditional mean replacement:} consiste nel rimpiazzare tutti i missing values con la media a condizione del fatto che sia presente un'ulteriore condizione posta su un secondo attributo. E' un po' più efficace degli altri elementi ma presenta anch'esso delle criticità.
	
	\item \textbf{Most probable:} consiste nell'eseguire una regressione sul nostro dataset utilizzando un altro attributo. In questo caso è possibile usare deei modelli di regressione anche molto complessi ed è possibile lavorare anche con dati qualitativi.
\end{itemize}

E' ora chiaro che il confine tra l'esplorazione dati e la modellizzazione degli stessi è molto sottile.

\subsection{Data Preprocessing}

Il preprocessing è un'area dell'analisi dati che consiste nel \textbf{rendere i dati più fruibile per una analisi dati.}

\subsubsection{Aggregation}
\textit{L'aggregazione consiste nel combinare due o più record in un unico oggetto.}
Ci sono diversi vantaggi ottenibili nell'aggregare i dati, in particolare ne elenchiamo alcuni:

\begin{itemize}
	\item Dataset più piccoli: durante un'analisi dati abbiamo bisogno di usare il minor quantitativo di memoria e di tempo, l'aggregazione in particolare diminuisce il tempo di esecuzione di un algoritmo.
	\item Cambiamento di scopo: ci permette di avere una visione più ampia dei dati qualora cambiassimo lo scopo della nostra nalisi in corso d'opera.
	\item Varianza ridotta: gli attributi calcolati su record aggregati sono più stabili rispetto a quelli associati ai records nativi, questo per un effetto statistico.
\end{itemize}


\subsubsection{Sampling}
In molti casi avere una quantità enorme di dati può essere deleterio dal punto di vista computazionale, questo perché bisognerebbe usare algoritmi più semplificati in modo che la computazione possa essere svolta su un maggior numero di dati. Ciò che dobbiamo preferire è invece usare un algoritmo migliore su un dataset ridotto. Per questo entra in gioco il concetto di \textbf{campionamento.}

Il problema diventa quindi: \textit{quando un campione è rappresentativo?}

\begin{defn}
	Un campione si dice \textbf{rappresentativo} quando ha approssimativamente le stesse proprietà del dataset di partenza.

\end{defn}

Dobbiamo quindi trovare uno schema che ci permetta di scegliere \textit{con grande probabilità} dei campioni rappresentativi. In questo caso il problema si riduce a trovare le appropriate:
\begin{itemize}
	\item Dimensioni del campione.
	\item Tecniche di campionamento.
\end{itemize}
Esistono moltissime tecniche di campionamento ma di seguito  mostreremo solo le più basilari:

\begin{itemize}
	\item \textbf{Simple Random Sampling:} Ogni record del dataset ha la stessa probabilità di essere incluso nel campione. Tale record può essere rimosso o meno dal dataset di partenza. Quando i campioni sono molto piccoli rispetto al dataset di partenza la rimozione o meno genera due campioni che sono molto simili.
	
	Questo metodo fallisce quando il datset consiste di attributi qualitativi in modo che i possibili valori che possono avere hanno frequenze fortemente diverse.
	
	\item \textbf{Stratified Sampling:} Vengono presi record in modo che all'interno del campione vengano rispettate le proporzioni tra gli attributi presenti nel dataset di partenza.
\end{itemize}

Una volta scelta la tecnica di campionamento dobbiamo occuparci di scegliere la grandezza del campione. Ovviamente tenere un campione di grande ampiezza aumenta la probabilità che un campione sia rappresentativo, di contro elimina la maggior parte dei vantaggi computazionali di avere un campionamento. Avere un campione troppo piccolo invece manda in contro al rischio di eliminare pattern potenzialmente importanti o addirittura di mantenere pattern erronei. La giusta dimensione del campionamento va scelta in base al nostro dataset di riferimento effettuando delle prove.

\subsubsection{Dimensionality reduction}

In diversi casici capiterà di analizzare dataset caratterizzati da un gran numero di attributi. Diminuirne il numero porta a diversi vantaggi:

\begin{itemize}
	\item Molti algoritmi lavorano se la dimensionalità è minore.
	\item L'interpretabilità del modello implementato aumenta perché dipende da un numero minore di attributi.
	\item La rappresentazione grafica dei dati è facilitata.
	\item L'ammontare di tempo e memoria diminuisce drasticamente.
\end{itemize}

Contrariamente a ciò che ci si aspetterebbe \textit{avere una dimensionalità maggiore dei dati non implica un aumento delle performance.}
Mostriamo questo dato con il sgeuente grafico.

\begin{figure}[H]
	\centering
	\includegraphics[height=0.35 \linewidth]{introduction/pict/performance.png}
	\caption{Andamento delle performance di un algoritmo in funzione della dimensionalità del dataset.}
\end{figure}

Molte delle tecniche usate per ridurre le dimensionalità sfruttano tecniche prese dall'algebra lineare per proiettare i dati da uno spazio dimensionale maggiore ad uno spazio dimensionale minore.

La \textbf{Principal Component Analysis (PCA)} trova nuovi attributi in modo che:
\begin{itemize}
	\item Siano \textit{combinazione lineari} degli attributi di partenza.
	\item Siano \textit{mutualmente ortogonali.}
	\item Catturino il \textit{massimo ammontare di variabilità} nei dati.
\end{itemize}

In particolare la \textbf{Singular Value Decomposition (SVD)} è una tecnica di algebra lineare correlata alla PCA ed è largamente usata per ridurre la dimensionalità dei dataset.

\begin{defn}
	Si chiama \textbf{binarizzazione} la tecnica di trasformazione di attributi continui e categorici in uno o più attributi binari
\end{defn}
Per farlo associamo $k$ possibili valori a $k$ valori interi nell'intervallo $[0, k-1]$. Successivamente trasformiamo questi k interi in un numero binario, come sappiamo il numero di cifre necessarie per rappresentare un numero intero son le sguenti:
\[s = [\log_{2}k]\] 
 In questo caso è necessario introdurre un attributo binario per ogni attributo categorico con cui il dato può manifestarsi. Ovviamente si è più interessati ai casi in cui si manifesta il valore 1 perché indica la presenza di tale attributo.
 
 Quando invece abbiamo a che fare con attributi durante un'analisi di classificazione o di associazione ricorrdiamo alla tecninca di \textbf{discretizzazione.} Come facilmente intuibile la miglior discretizzazione dipende dal tipo di algoritmo che sto utilizzando.

 La discritezzazione può essere:
 
 \begin{itemize}
 	\item \textbf{Supervisionata.}
 	\item \textbf{Non-supervisionata.}
 \end{itemize}
La \textit{discritezzizazione supervisionata} utilizza ulteriori informazioni (attributi di classe) per discretizzare gli attributi. Questa tecninca divide i punti in modo che qualche misura di \textit{purità} sia massimizzata. La misura di purità più spesso applicata è l'\textbf{entropia.}
 	
 	\[e_{i} = - \sum_{k = 1}^{K}p_{ki}\log_{2}p_{ki}\]
 	
Per come è costruita si ha che:
\begin{itemize}
	\item Se contiene solo record di una data classe $\implies e_{i} = 0$, massima purità. 
	\item Se contiene egualmente spesso tutte le class $\implies e_{i} = \max$, minima purità. 
\end{itemize}

La discretizzazione supervisionata basata sull'entropia permette di trovare i punti di divisione degli attributi \textit{continui} tali che l'entropia totale sia minimizzata.

\begin{defn}
	Si definisce entropia totale la seguente espressione:
	\[ E = \sum_{i = 1}^{n}w_{i}e_{i}\] 
	In cui:
	\[w_{i} = \frac{m_{i}}{m}\]
	E le variabili sono così definite: \begin{itemize}
		\item $n$: numero di intervalli.
		\item $m$: numero di record.
		\item $m_{i}$: numero di record nell'intervallo i-esimo.
	\end{itemize}
\end{defn}


 	
La \textit{discretizzazione non-supervisionata} non utilizza alcuna informazione eccetto il valore dell'attributo continuo da discretizzare. Possono essere a loro volta divisi in due categorie:
 	\begin{itemize}
 		\item \textbf{equal width unsupervised discretization:}  gli intervalli in cui viene discretizzati hanno tutti la stessa ampiezza
 		\item  \textbf{equal frequency unsupervised discretization:} gli intervalli in cui viene discretizzato hanno approssimativamente la stessa frequenza.
 	\end{itemize}
 
 
 Può succedere però che gli attributi categorici abbiano troppi valori. Se gli attributi  in questione sono ordinali allora si applicano le tecniche viste in precedenza, se invece sono nominali allora bisogna trovare un altro approccio. In questo caso possiamo infatti raggruppare i valori solo se il raggruppamento si traduce in un miglioramento nelle performance di classificazione o nel raggiungimento di qualche obiettivo del trattamento dati. 
\subsubsection{Variable transformation}

\begin{defn}
Una trasformazione di variabile sè una trasformazione che è applicata a tutti i valori di quella variabile.
\end{defn}

Ci sono due tipi di trasformazione di variabile:
\begin{itemize}
	\item \textbf{Funzioni semplici}: una semplice funzione matematica è applicata a tutti i valori di una variabile individualmente. Bisogna stare attenti all'ordine in cui la applichiamo e soprattutto a come si comporta la funzione per valori negativi e vicini allo 0.
	\item \textbf{Standardizzazione:} trasforma tutto il dataset in modo che acquisiscano una particolare proprietà.
	\[Z = \frac{x - \mu}{\sigma}\]
	E' molto importante perché il nuovo dataset avrà $\mu = 0$ e $\sigma = 1$ per come l'ho costruito. In questo modo la somma di diversi attributi continui permette ad uno o a pochi attributi di prendere grandi valori e di dominare il nuovo attributo somma. E' molto utile quindi perché fa saltare immediatamente all'occhio la presenza di outlier.
	
	In questo caso la media è sostituita dalla mediana, e la deviazione standard è sostituita dalla AAD che ricordiamo essere definita come:
	\[\sigma_{x} = \frac{1}{m} \sum_{i = 1}^{m}|x_{i}- \mu|\]
\end{itemize}

\clearpage
\section{Classification}

\subsection{Introduction (*)}
Entriamo nel dettaglio della componente di validazione nei modelli di classificazione supervisionata.
La cosa più importante è \textbf{capire cosa stiamo facendo} in quanto quando si applicano algoritmi di machine learning è facile perdersi.

Nel solito dataset dei churn vogliamo determinare se un cliente abbandonerà o meno il nostro servizio; per farlo, cerchiamo di utilizzare una particolare combinazione di attributi. Bisogna innanzitutto vedere di che tipo sono le variabili presenti nel nostro database.

Lo scopo è di essere  in grado sia di prevedere quando un cliente abbandonerà un determinato servizio, che di capire \textbf{le motivazioni} che l'hanno spinto a farlo. Questo tipo di compiti può essere portato a termine con l'utilizzo di un modello di \textbf{classificazione}. Un modello di classificazione è un modello che sfrutta alcuni attributi del dataset (\textbf{attributi esplicativi}) per prevedere un valore di un altro attributo (\textbf{attributo di classe}).

\begin{figure}[H]
	\centering
	\includegraphics[width= \linewidth]{classification/pict/class_model.png}
	\caption{processo di classificazione}
\end{figure}

Forniamo di seguito delle definizioni utili per poter parlare più nel dettaglio dei metodi di classificazione:
\begin{defn}
	Si definiscono \textbf{variabili esplicative} le variabili in input.
\end{defn}

\begin{defn}
	Si definiscono \textbf{variabili di classe}	Le variabili in output.
\end{defn}

\begin{defn}
	 Si definisce \textbf{modello di classificazione} un modello in grado di risolvere un problema di classificazione; i modelli di classificazione si dividono in:
	\begin{itemize}
		\item \textbf{Modello descrittivo}: serve come strumento di spiegazione per distinguere tra oggetti di classi diverse
		\item \textbf{Modello predittivo}: predice la classe di un record sconosciuto, pu\`o essere visto come una \underline{scatola nera} che asssegna una label di una classe al record sconosciuto
	\end{itemize}
\end{defn}

\begin{defn}
	Si definisce \textbf{classificatore} l' approccio sistematico per costruire un modello di classificazione su un dataset.
\end{defn}
La costruzione di un classificatore segue il seguente processo:
\begin{figure}[H]
	\centering
	\includegraphics[height=0.6 \linewidth]{classification/pict/class_process.png}
	\caption{Processo di creazione della classificazione}
\end{figure}

\begin{defn}
	Si definisce \textbf{training set}  l'insieme dei record in cui tutti i valori degli attributi sono noti.
\end{defn}

\begin{defn}
	Si definisce \textbf{test set} l'insieme dei record i cui valori dell'attributo di classe sono sconosciuti (o presunti tali).
\end{defn}

Si utilizza quindi un algorimo di learning (\textbf{Learner}) sul training set, passaggio definito come \textbf{Classification Model Learning}. 

\begin{defn}
	Si definisce \textbf{Inducer} l'output di questa operazione.
\end{defn}

Il ruolo dell'inducer è quello di predire il valore dell'attributo di classe per il test set. 
La chiave di tutto è la scelta dell'algoritmo learner per l'apprendimento; la scelta degli attributi esplicativi è fondamentale, alcuni di essi possono, infatti, essere solo ridondanti nel processo e generare solo rumore. Tramite l'analisi delle performace si possono valutare queste due scelte. 

Bisogna valutare le performance del modello inducer. Uno dei modi per analizzare se è efficace è la \textbf{matrice di confusione}. 

Nelle righe vi sono i veri valori delle classi, nelle colonne i valori predetti dall'inducer. Il numero di righe è uguale al numero di colonne e corrispondono al numero di classi da predire.

\begin{figure}[H]
	\centering
	\includegraphics[height=0.25 \linewidth]{classification/pict/matrconf.png}
	\caption{modello di matrice di confusione}
\end{figure}
All'interno sono identificati i risultati dei test e sono così classificati:
\begin{itemize}
	\item \textbf{TN - vero negativo}: num. di record dove AC=-1  che sono stati corretamente predetti IP=-1
	\item \textbf{FN - falso negativo}: num. di record dove AC=+1  che sono stati erroneamente predetti IP=-1
	\item \textbf{TP - vero positivo}: num. di record dove AC=+1  che sono stati corretamente predetti IP=+1
	\item \textbf{FP - falso positivo}: num. di record dove AC=-1  che sono stati erroneamente predetti IP=+1
\end{itemize}

\begin{defn}
	La misura più utilizzata come metrica di performance è l'\textbf{accuratezza}:
	
	\[accuracy = \frac{\sum_{i = 1}^{n} diagonal}{\sum_{i=1}^{n}elements} = \frac{\#CorrPrediction}{\#Prediction} = \frac{TN + TP}{TN + TP + FN + FP}\] 
	
	La misura complementare che troveremo indicata è quella dell'errore:
	
	\[error = \frac{FN + FP}{TN + FN + FP + TP} =  1 - accuracy\]
\end{defn}

\subsection{Tecniche di classificazione}

Una tecnica di classificazione è un modo sistematico di aggredire un dataset e costruire i modelli di classificazione, in particolare possiamo dividerle in 4 macro categorie:
\begin{itemize}
	\item \textbf{Euristiche}: ispezionano il suo vicinato come ad esempio: Decision Trees, Random Forest, Nearest Neighboor
	\item \textbf{Regression Based}: usa la probabilità condizionata parametrica, ad esempio: regressione logica
	\item \textbf{Separazione}: partiziona lo spazio degli attributi, fa riferimento alle Support Vector Machine e alle Artificial Neural Network
	\item \textbf{Probabilistici}: usano la formula di Bayes (Naive Bayes ecc...).
\end{itemize}

Forniremo una overview di tutte queste categorie senza trattarle nel dettaglio; partiamo dalla trattazione dei modelli \textbf{Euristici}.

\subsubsection{Decision Tree}
I modelli decision tree sono modelli di tipo euristico. 

Un \textbf{albero di decisione} ha una rappresentazione grafica precisa in cui gli elementi principali sono: nodi e archi. Il \textbf{nodo} rappresenta un sottoinsieme del dataset, gli \textbf{archi} sono usati, invece, per modellare gli output di modelli diversi di dataset.

\begin{figure}[H]
	\centering
	\includegraphics[height=0.7 \linewidth]{classification/pict/decision_tree.png}
	\caption{Modello decision tree}
\end{figure}

Gli elementi principali da cui è costituito sono:
\begin{itemize}
	\item \textbf{Root node} (nodo radice): non ha archi in ingresso ma può averne più di due in uscita
	\item \textbf{Internal nodes} (nodi interni): hanno un solo arco in ingresso e almeno due in uscita
	\item \textbf{Leaf} (nodi foglia o terminali): sono nodi interni senza archi in uscita e con esattamente un arco in ingresso.
\end{itemize}

Ogni record viene classificato partendo \textbf{dall'alto} (radice) fino \textbf{al basso} (nodi foglia).

Leggo il primo nodo (radice) in cui è presente l'indicazione di un attributo: nell'esempio se 'day charge' sia $\le 41,665$ o  $>$ e divide l'albero in due nodi. Se la risposta è vera allora mi muovo in un nodo altrimenti nell'altro. All'interno del nuovo nodo ho di nuovo la valutazione di una variabile, e proseguo così finché raggiungo un nodo foglia.
Quando si arriva ad una foglia devo fornire una risposta, ovvero conto la classe più frequente e rispondo al chiamante con essa (es. churn = n).

Il decision tree può essere usato per attributi nominali, ordinali così come per intervalli numerici e coefficienti.
 
Diverse misure possono essere usate per selezionare la politica di node splitting ottima: \textit{entropia}, \textit{indice di Gini} e il \textit{classification error}. Esso comunque dipende anche dal tipo di attributo: \textit{binario}, \textit{nominale} e \textit{continuo}. 

E' un modello che risponde sempre \textit{la classe più frequente ed è quindi inutile quando ho delle classi particolarmente sbilanciate.} Sono evidentemente test univariati, ciò che faccio è costruire iper parallelepipedi del nostro dataset. Vanno a definirsi delle rette per il cambio di classificazione: \textbf{decision boundary}.

\begin{figure}[H]
	\centering
	\includegraphics[height=0.5 \linewidth]{classification/pict/decision_boundary.png}
	\caption{Decision boundary}
\end{figure}

Il risultato di queste rette fa la differenza sulla capacità di evidenziare dove siano presenti elementi di una certa classe, devo quindi imparare a posizionare questi iperpiani: \textbf{l'obiettivo è partizionare il mio spazio in iperpiani massimizzando l'accuratezza.}

Non c'è alcuna ragione per cui io non possa usare degli splitting multipli e non solo su binari, posso farlo anche su valori nominali.

Vi sono poi diversi modelli derivati dal decision tree, come il Random Forest.

\textbf{Random forest}: è un comitato di alberi di decisione. Usa degli attributi che sono in generale sottoinsiemi degli attributi, ogni albero può avere un sottoinsieme differente di attributi. Ogni albero usa attributi in modo casuale. Ogni albero apprende a modo suo e il random forest in base a dei parametri (regione dello spazio) decide a quale albero ascoltare per la decisione.

\subsubsection{Regressione Logistica Binomiale}
La regressione logistica è un metodo di classificazione basato sulla regressione.

Assumiamo che la variabile di classe Y sia compresa tra $\{0,1\}$, allora la regressione logistica binaria calcola a \textit{posteriori} la probabilità che Y dia il valore dell'input ovvero le variabili esplicative.
Servono per risolvere problemi di regressione binaria a diversi livelli. E' applicabile ad attributi continui e con certe accuratezze anche ad attributi nominali.

Si assume che l'attributo di classe sia tale che $Y = \{0,1\}$;  il classificatore a regressione logistica binomiale calcola a posteriori la probabilità che $Y$ assuma il valore di un input esplicativo $\underline{X}$. Si cacola come segue:

\[P(Y = 0 | \underline{X}= \underline{x}) = \frac{1}{1+exp(\underline{x} \cdot \underline{w})}\]

\[P(Y = 1 | \underline{X}= \underline{x}) = \frac{exp(\underline{w} \cdot \underline{x})}{1+exp(\underline{w} \cdot \underline{x})}\]

dove $\underline{w}$ è chiamato \textit{vettore parametro.}

\begin{figure}[H]
	\centering
	\includegraphics[height=0.5 \linewidth]{classification/pict/regr_logistic.png}
	\caption{Esempio di regressione logistica}
\end{figure}

In questo modo vengono calcolate le probabilità di appartenere ad una certa classe.	 

\subsubsection{Support Vector Machines}
La tecnica Support Vector Machines è un metodo di classificazione con separazione. Lo scopo è separare o "apprendere" classi che vogliamo classificare. 

Consideriamo uno spazio bidimensionale in cui sono rappresentati $m$ record dove due attributi continui vengono misurati: $D = \{(\underline{x}_1, y_1),...,(\underline{x}_m, y_m)\}$ in cui $\underline{x}_i \in R^2$ e $y_i \in \{-1, +1\}$

L'idea è quella di tracciare una retta (se ho due sole classi) per cui definisco l'area di appartenenza di una o dell'altra classe. 

\begin{figure}[H]
	\centering
	\includegraphics[height=0.4 \linewidth]{classification/pict/svm.png}
	\caption{Esempio di una possibile retta di separazione}
\end{figure}

La retta in questione è definita dalla seguente equazione:

\[\underline{w} \cdot \underline{x} + b = w_1 x_1 + w_2 x_2 + b = 0\]

Per orientare la retta utiliziamo il vettore $\underline{w} = [w_1,w_2]$ oppure $b$. ($\underline{w}$ fa ruotare, $b$ fa traslare)\\
se la retta esiste allora l'insieme delle istanze è \textit{linearmente separabile}.

\textit{Problema}: possono esserci più (anche infinite) rette utilizzabili per effettuare la separazione, quale devo scegliere? Perché non basta trovare una retta che funziona, ma la retta migliore in quanto l'algoritmo dovrà essere testato su dati nuovi. Si crea una sorta di area grigia nella quale non so dire esattamente quale delle due classi categorizzare (\textbf{Linear Decision Boundary}).

\begin{figure}[H]
	\centering
	\includegraphics[height=0.4 \linewidth]{classification/pict/svm_rette.png}
	\caption{Diverse rette soluzioni dello stesso problema}
\end{figure}

Bisogna sostanzialmente trovare una retta che \textbf{massimizza il margine di errore}, l'\textit{optimal linear decision boundary}. 

\begin{figure}[H]
	\centering
	\includegraphics[width= \linewidth]{classification/pict/svm_boundary.png}
	\caption{Boundary}
\end{figure}

La retta $B_1$ è nettamente preferita rispetto alla retta $B_2$ in quanto ha un margine di supporto $\delta$ maggiore. 

Naturalmente per $n$ attributi bisogna trovare l'\textit{iperpiano} ottimale per dividere un determinato insieme di istanze. Matematicamente parlando si cerca di massimizzare il margine $\delta = 1 / |w|^2$, dove la retta ha la seguente impostazione: $\bar{w} \cdot \bar{x} + b = 0$. Le rette ai confini del margine sono fissate a $\bar{w} \cdot \bar{x} + b = 1$ e $\bar{w} \cdot \bar{x} + b = -1$. 

L'argomento della retta è in 2 dimensioni, però dopo aver applicato la funzione $h(\underline{x})$ quella retta diventa un piano che va a $-1$ da un lato e a $+1$ dall'altro.

Il training di una \textbf{SVM Linear Hard-margin} avviene formulando e risolvendo il seguente problema matematico: 

\[ \min_{\underline{w},b} \frac{1}{2}\underline{w} \cdot \underline{w}^T \]
\qquad s.t.
\[ y_i (\underline{w} \cdot \underline{x}_i + b) \ge 1 \quad \forall i = 1, ..., m\]

È un problema di programmazione quadratica con vincoli lineari i quali devono essere risolti con tecniche numeriche speciali.

Per trovare la retta devo minimizzare l'inverso del margine $\delta$; per farlo bisogna imporre dei vincoli per ogni attributo. In particolare,  se $\underline{w}$ e $\underline{x}$ sono \textit{concordi in segno} (positivo o negativo) allora classifica perfettamente perché il risultato è $>= 1$. Questi vincoli garantiscono che tutti i casi del dataset siano classificati correttamente e tra tutti i casi che classificano correttamente scelgo quello che massimizza il margine. 

Non risolveremo questa formula di ottimizzazione in modo diretto, ma la sua formulazione \textbf{duale}. In ogni caso questa formulazione funziona bene se il problema è  \textit{linearmente separabile}.

Nei casi in cui il problema\textbf{non} è linearmente separabili, non esiste \textbf{mai} una retta in grado di separare correttamente le classi. In questi casi la formulazione precedente non ammette soluzione, perché per alcuni dati i vincoli non ammettono soluzione. 

Si introduce allora la \textbf{Linear Soft-margin}:

\[ \min_{\underline{w},b,\underline{\epsilon}} \frac{1}{2}\underline{w} \cdot \underline{w}^T + \Delta \sum_{i=1}^{m} \epsilon_i \]
\qquad s.t.
\[\forall_{i=1}^m :  y_i (\underline{w} \cdot \underline{x}_i + b) \ge 1 - \epsilon \] \[\forall_{i=1}^m : \epsilon \ge 0\]

Le $\epsilon$ devono essere non negative (variabili di slack), se utilizzo questo parametro  per ammettere un errore allora esiste almeno una retta che risolve il problema di ottimizzazione. Ho sostanzialmente \textit{rilassato} il problema di ottimizzazione, in particolare i vincoli. Graficamente parlando ciò si traduce nel traslare degli elementi di una classe diversa dalla regione di appartenenza verso la regione della classe di quell'elemento.
E' molto importante notare che vettori di supporto sono quelle osservazioni che sono sul bordo del margine

\textbf{Problema}: La separazione si presta ad essere eseguita solo con una funzione \textbf{non lineare}.
 
\textbf{Soluzione}: Si va a cercare una trasformazione che porti dallo spazio originale in uno spazio delle features in cui posso applicare una separazione lineare. Nel nuovo spazio sono in grado di separare il nuovo dataset in due diverse classi (vedi immagine che segue). 

\begin{figure}[H]
	\centering
	\includegraphics[width=0.9 \linewidth]{classification/pict/svm_nonlinear.png}
	\caption{Applicazione di una separazione non lineare}
\end{figure}

In questo modo posso sfruttare tutta la metodologia precedente ma in uno spazio "controllato". Bisogna trovare però una trasformazione $\phi(x)$ che mappi X in F.
La nuova Linear Deision Boundary nello spazio delle feature F è definito dalla seguente equazione: 
\[\underline{w} \cdot \phi(\underline{x}) + b = 0\]

Il modello diventa:

\[ \min_{\underline{w},b} \frac{1}{2}\underline{w} \cdot \underline{w}^T \]
\qquad s.t.
\[ y_i (\underline{w} \cdot \phi(\underline{x}_i) + b) \ge 1 \quad \forall i = 1, ..., m\]

Per l'algoritmo di apprendimento utilizziamo sostanzialmente delle funzioni kernel che sono del tipo:
 \[K(\underline{u}, \underline{v}) = \phi(\underline{u}) \cdot \phi(\underline{v})\]
 
 Queste funzioni kernel sono delle funzioni di similarità calcolate nello spazio attributi originale di $x$ e sono riferite alla funzione kernel.

\subsubsection{Multi-Layer Perceptron o Artificial Neural Network}
Il Multi-Layer Perceptron (MLP) è una tecnica di classificazione che si basa sulla separazione dello spazio degli attributi; queste tecniche sono oggi molto utilizzate per il deep learning.
\begin{defn}
 	Si definisce \textbf{MLP Multi-Layer Perceptron} un modello che consiste in neuroni artificiali che comunicano in modo unidirezionale, dall'input X alla variabile di classe (volendo ci possono essere più neuroni di ouput).
\end{defn}

\begin{figure}[H]
	\centering
	\includegraphics[width=0.8 \linewidth]{classification/pict/mlp.png}
	\caption{Esmpio di MLP}
\end{figure}

In figura vi sono 3 neuroni di parametri continui a cui viene aggiunto un quarto neurone che calcola una funzione. Ad ogni input nel neurone è associato un peso $w$.
Il neurone calcola una \textbf{combinazione lineare} tra gli input ed il peso dato ad esso, il j-esimo neurone calcola: 

\[ y_j = f(\sum_{i=1}^{n} w_{i,j} \cdot x_i - \theta_j)\] 

Ad esempio il quarto neurone in figura vale:  
\[z_4 = w_{1,4} \cdot x_1 + w_{2,4} \cdot x_2 +w_{3,4} \cdot x_3\]

A questo calcolo viene posto una valore di soglia pari a $\theta$ applicato alla combinazione lineare. Successivamente,  il neurone risponde con un valore pari all'applicazione di una funzione di attivazione rispetto all'input meno il threshold, nel nostro caso $f(z_4-\theta_4)$. 

\begin{defn}
	Si definisce \textbf{funzione di attivazione} la funzione applicata ad un neurone che restituisce il valore che verrà portato ad un altro strato con cui questo neurone comunica.
\end{defn}
Storicamente le funzioni applicazione sono la tangente iperbolica e la funzione logistica (intervallo tra -1 e 1 e tra 0 e 1). Oggi vengono utilizzate delle funzioni non derivabili più complesse, come funzioni \textbf{RELU} (Retify Linear Unit) che assumono valore 0 nel semiasse negativo e 1 nel semiasse positivo.
\\

Vi sono 3 tipi di neuroni: di input, di output e nascosti:
\begin{itemize}
	\item \textbf{Input} sono collegati con le variabili esplicative e con \textit{ciascun} nodo nascosto.
	\item \textbf{Nascosti} (o hidden) ricevono i segnali dai neuroni di input e il segnale viene propagato a quelli di ouput.
	\item \textbf{Output} sono associati alla variabile di classe e ricevono i segnali da visualizzare.
\end{itemize}

Ogni neurone di input è connesso con ogni neuroni di strato nascosto  ed ogni nodo di strato nascosto comunica con il nodo di output, rete \textbf{fully-connected}. Ogni arco ha associato un peso e ogni nodo ha un valore di soglia $\theta_j$ e una funzione di attivazione. 

\begin{figure}[H]
	\centering
	\includegraphics[height=0.5 \linewidth]{classification/pict/mlp_struct.png}
	\caption{Modello MLP semplice}
\end{figure}

Per la disposizione dell'architettura ho diverse scelte da fare: quanti neuroni usare, quanti nello strato nascosto, quanti strati nascosti usiamo, che funzione di attivazione, ecc.

Determinare l'architettura della nostra rete è fondamentale per avere buone performance. Posso pensare di aggiungere un \textbf{livello} di neuroni nascosti. Non vi sono vincoli rispetto a comunicare saltando livelli. \textbf{Non} si può però comunicare all'indietro, infatti queste reti sono chiamate \textbf{feed-forward neural network} (possono possedere fino a centinaia di strati nascosti).

\begin{figure}[H]
	\centering
	\includegraphics[width=0.9 \linewidth]{classification/pict/mlp_esempio.png}
	\caption{Esempio MPL con 2 strati nascosti}
\end{figure}

Per migliorare l'apprendimento si possono propagare le valutazioni fatte all'output per tutti i nodi risalendo fino all'input modificando i pesi degli archi. Questa cosa funziona bene e ha senso con le RELU.

Il problema MLP. non ha soluzione oggi, pertanto si procede in modo \textbf{empirico}, non è ancora possibile, infatti, arrivare ad una soluzione ottima, non si riesce a capire se si ha raggiunto il massimo/minimo globale ma si cerca quella che dà risultati accettabili e migliori di altri. Il numero di nodi, archi o livelli nascosti è quindi scelto in base all'esperienza.


\subsubsection{Classificatori bayesiani}
I \textbf{classificatori probabilistici} calcolano la probabilità condizionata usando il \textbf{teorema di Bayes} e cercano di capire il valore della classe attributo dalle altre variabili. 

\begin{defn}

La \textbf{formula di Bayes} calcola la probabilità che un evento accada dati altri eventi: 

\[P(Y|\underline{X}) = \frac{P(\underline{X}|Y) \cdot P(Y)}{P(\underline{X})}\]

Dove:
\begin{itemize}
	\item $P(Y)$ è la probabilità dell' attributo di classe.
	\item $P(\underline{X}|Y)$ la verosimiglianza di un vettore di attributi dato l'attributo di classe.
	\item $P(\underline{X})$ probabilità dell'evidenza (nel senso di certezza).
	\item $P(Y|\underline{X})$ probabilità a  posteriori dell'attributo di classe  dato il vettore di attributi esplicativi.
\end{itemize}

Vediamone un esempio, assumiamo di avere:
\begin{itemize}
	\item $Y$ variabile binaria di classe $\{-1,+1\}$
	\item $\underline{X}$ variabile bianria esplicativa $\{Male, Female\}$
\end{itemize}
\end{defn}
Assumiamo di avere le seguenti probabilità: 
\[P(Y) = (0.3,0.7) \quad P(\underline{X}|Y=-1) = (0.2,0.8) \quad P(\underline{X}|Y = +1) = (0.9,0.1)\]

Vogliamo classificare $\underline{X} = Male$ tramite la formula di Bayes:

\[P(Y = -1 | \underline{X} = M) = \frac{P(\underline{X} = M | Y=-1) \cdot P(Y=-1)}{P(\underline{X}= M)} = \frac{0.06}{P(\underline{X} = M)}\]

\[P(Y = +1 | \underline{X} = M) = \frac{P(\underline{X} = M | Y=+1) \cdot P(Y=+1)}{P(\underline{X}= M)} = \frac{0.63}{P(\underline{X} = M)}\]

In questo caso quindi la $Y$ è più probabile che abbia valore $+1$ se $\underline{X} = Male$.

Assumiamo ora che il numero di variabili esplicative aumenti all'aumentare \textbf{n} ed esse sono binarie, così come l'attributo di classe, allora avremo bisogno di conoscere i seguenti parametri:

\[\theta_{ki} = P(\underline{X} = x_k | Y = y_i) \qquad k \in \{1, ..., 2^n\}, y_i \in \{-1,+1\}\]

Quando il numero di variabili esplicative cresce, le devo binarizzare quindi avrò $2^n$ parametri:

\begin{table}[H]
	\centering
	\begin{tabular}{|c|c|c|c|c|c|c|}
		\hline
		\textbf{n} & 1 & 2 & 3 & 10 & 20 & 30 \\
		\hline
		\textbf{$\#$ parametri}& $2$ & $4$ & $8$ & $1,024$ & $1,048,576$ & $1,073,741,824$  \\
		\hline
	\end{tabular}
\end{table}

Per risolvere il problema si usa l'\textbf{assunto della condizione di indipendenza}.

\subsubsection{Naive bayes}

\begin{defn}
Dati gli attributi X, Y, Z, diremo che X è \textbf{indipendente condizionatamente} da Y dato Z, se e solo se la probabilità di X è indipendente dal valore dell'attributo Y una volta che Z sia noto, formalmente: 

\[\forall i,j,k P(X=x_i|Y=y_j, Z=z_k) = P(X=x_i|Z=z_k)\]
\end{defn}

Assumendo questa espressione per le variabili utilizzate si riduce enormemente il numero di parametri da calcolare; il \textbf{Naive Bayes} assume indipendenza condizionale e ci permette di calcolare la probabilità a posteriori dell'attributo classe dati gli attributi esplicativi come segue: 

\[P(X_1, ..., X_n|Y) = \prod_{i=1}^n P(X_i|Y)\]

\[2 \cdot (2^n-1) \rightarrow 2 \cdot n\]

Il \textbf{Classificatore Naive Bayes} calcola la probabilità a posteriori dell'attributo di classe nel modo seguente:

\[P(Y=y_k|X_1,...,X_n) = \frac{P(Y=y_k)\cdot \prod_{i=1}^{n}P(X_i|Y=y_k)}{\sum_{j}P(Y=y_j) \cdot \prod_{i=1}^{n}P(X_i|Y=y_j)} \]

Il record di variabili viene etichettato con il valore di classe che massimizza la probabilità a posteriori

\[arg\max_{y_k} P(Y = y_k | X_1,...,X_n)\]

Il modello Naive Bayes fa parte dei modelli \textbf{grafico-probabilistici}. Esiste un' intera famiglia di modelli di questo tipo: reti bayesiane dinamiche, ecc... non andremo nel dettaglio.

\begin{figure}[h!]
	\centering
	\includegraphics[height=0.3 \linewidth]{classification/pict/naivebayes.png}
	\caption{Rete naive bayes}
\end{figure}

A fianco del risultato output comunque è bene fornire la probabilità con il quale si è ottenuto il risultato, in modo da dare una \textit{misura di affidabilità.} 

Il classificatore Naive bayes può essere applicato ad attributi categorici (nominali e ordinali), numerici (intervalli e tassi). Ad ogni attributo numerico è associata una densità di probabilità condizionale di classe normale:

\[P(X_i = x_i | Y = y_k) = \frac{1}{\sigma_{ik}\sqrt{2 \cdot \pi}}exp(-\frac{(x_i-\mu_{ik})^2}{2 \sigma_{ik}^2})\] 

i cui parametri sono: 
\[\mu_{ik} = E[X_i | Y = y_k] \qquad \sigma_{ik}^2 = E[(X_i - \mu_{ik})^2 | Y = y_k]\]

\textbf{NB:} solitamente si combinano attributi categorici (nominali e ordinali) con attributi numerici (intervalli e tassi).

Il modello Naive bayes si comporta generalmente bene ma richiede \textit{una enorme premessa} per essere applicato correttamente, ovvero l' (\textbf{indipendenza condizionata}). Per ovviare a questo problema è stata creata una versione più flessibile mantenendo la stessa capacità computazionale.

\subsubsection{Reti bayesiane}
Il classificatore Naive bayes viene generalizzato dal modello di rete \textbf{Rete bayesiana} che è meno forzato dall'assunzione di indipendenza condizionata ma sfrutta il concetto di \textbf{Sparsity}. 

Un esempio di rete bayesiana è il seguente:

\begin{figure}[H]
	\centering
	\includegraphics[height=0.35 \linewidth]{classification/pict/networkbayes.png}
	\caption{Esempio di rete bayesiana generalizzata}
\end{figure}
Gli attributi esplicativi sono associati tramite i nodi, ma non solo quelli con un arco in entrata dalla radice; gli attributi esplicativi possono puntare ad altri attributi esplicativi.
Nel grafo \textbf{non} vi possono essere \textbf{cicli} in quanto un nodo successivo non può essere causa di un nodo precedente. Per ogni nodo bisogna specificare una tabella di probabilità condizionata rispetto al valore dei suoi genitori. In sostanza prendo in considerazione tutte le possibili configurazioni dei genitori e per ognuna do un output diverso. 

In questo modello gli attributi esplicativi non sono più assunti come indipententi dalla classe attributo.

Questo è un modello particolare di rete bayesiana che viene utilizzato per la classificazione. La cosa bella di questo modello è che anche con valori null si può comunque fare inferenza perché nativamente ha questa caratteristica. 

\textit{Nelle reti neurali non è possibile far computare un modello senza avere tutti gli input. }

\subsubsection{Tree-augmented Naive Bayes}
Vi sono diverse versioni di reti bayesiane, una di queste è il Tree-augmented Naive Bayes.
\begin{defn}
Si definisce \textbf{Tree-augmented Naive Bayes} una rete che oltre ad avere il nodo genitore Y può permettersi di avere un altro nodo genitore (sempre). 
\end{defn}

Se io addestro la rete senza nodo di classe allora ho un albero, dopo inserendovi il nodo Y fa inferenza (nel caso in figura vedi nodo $X_1$).

\begin{figure}[H]
	\centering
	\includegraphics[height=0.35 \linewidth]{classification/pict/treenaivebayes.png}
	\caption{tree-augmented Naive bayes}
\end{figure}

\`E molto potente per la feature selection, ovvero per la ricerca delle feature più significative. 
Ci sono altre modifiche possibili al Tree-augmente naive bayes, presenti il letteratura, come AODE, HNB, ecc... 

\subsubsection{Summary}
Per capire come \textit{nativamente} si differenziano i modelli di classificazione è presente la  tabella successiva. 

\begin{figure}[H]
	\centering
	\includegraphics[height=0.3 \linewidth]{classification/pict/class_tecniques.png}
	\caption{Comparazione tra i principali modelli nativi}
\end{figure}

In ogni caso, è possibile \textbf{trasformare} attributi esplicativi e di classe nominali applicando tecniche di classificazione non assegnate per attributi nominali. \`E da notare comunque che le tecniche di classificazione \textbf{non} sono l'opzione migliore quando bisogna predire un attributo \textbf{ordinale}.

\subsection{Performance Evaluation (*)}
La stima dell'accuratezza non è sufficiente da sola per comprendere la confidenza di un modello, in particolare, quando bisogna applicarlo a dati mai visti.
Bisogna rendersi conto del rischio di \textbf{overfitting} quindi magnificare alcuni risultati, e di \textbf{underfitting} ovvero non abbastanza accurati.

Vi sono due diversi tipi di \textit{errori}:
\begin{itemize}
	\item \textbf{Training error}: numero di record del training set mal classificato.
	\item \textbf{Generalization error}: errori su record no visti precedentemente (test set).
\end{itemize}
\textit{Un buon classificatore non deve fittare troppo bene il training set ma deve anche classificare accuratamente i record che non ha mai visto prima.}

\begin{defn}
	Si parla di \textbf{Overfitting del modello} quando un modello di classificazione ottiene delle performance molto alte sul training set ma ha un generalization error molto alto.
\end{defn}

\underline{Nella pratica}: non bisogna adattare troppo il modello sui dati di training ma bisogna puntare ad un buon compromesso tra l'errore fatto sul training set e quello fatto sul test set. 

Consideriamo il seguente esempio:
\begin{figure}[H]
	\centering
	\includegraphics[height=0.3 \linewidth]{classification/pict/overfitting.png}
	\caption{Esempio di overfitting}
\end{figure}
\begin{itemize}
	\item Linea {\color{blue}{blu}}: modello con poco training error (non overfitting)
	\item Linea {\color{orange}{arancione}}: modello con zero training error (overfitting)
\end{itemize}
Come si può notare il modello {\color{orange}{arancione}} si adatta troppo ai dati di training quindi c'è un forte rischio che non classifichi bene i dati di test. Al contrario il modello {\color{blue}{blu}} segue meglio la tendenza della distribuzione delle due classi di elementi.

Ora vediamo come si comportano i due modelli nel test set
\begin{figure}[H]
	\centering
	\includegraphics[height=0.3 \linewidth]{classification/pict/overfitting_test.png}
	\caption{Esempio di overfitting test}
\end{figure}
Ora si può notare come il modello {\color{blue}{blu}}: più alto training error e più basso generalization error) classifichi \textbf{più correttamente} rispetto al modello {\color{orange}{arancione}}: più basso training error e più alto generalization error).

\begin{defn}
Il fenomeno contrario è definito \textbf{Underfitting}, in cui sia il training error che il generation error sono elevati. Accade quando performance del training e test sono simili e basse.
\end{defn}
La complessità del modello scelto non è sufficiente per i dati che sto studiando. Non abbiamo utilizzato tutta la flessibilità che potevamo usare nella definizione del modello.

\begin{figure}[H]
	\centering
	\includegraphics[height=0.3 \linewidth]{classification/pict/underfitting_merge.png}
	\caption{Esempio di underfitting (sinistra)}
\end{figure}
Guardando agli esempi in figura, il modello {\color{blue}{blu}} classifica erroneamente 6 record, per questo non è un buon modello. Invece, il modello {\color{orange}{arancione}} fitta meglio il training ed risponde meglio anche al test (1 errore), in questo caso però bisogna stare attenti a non incappare in overfitting.
 
La \textbf{soluzione ottimale} la si evince tenendo conto di quello che succede sia nei dati di training che nei dati di test.

\textit{\`E impossibile avere una misurazione certa sugli errori del modello, si può solo avere una stima. }

\subsubsection{Performance di un modello di classificazione}

In una analisi di classificazione l'obiettivo è sviluppare un modello che dia una migliore classificazione possibile. Viene valutata in termini di:
\begin{itemize}
	\item \textit{Accuracy}
	\item \textit{Speed}
	\item \textit{Robustezza}
	\item \textit{Scalabilità}
	\item \textit{Interpretabilità} (tema più dibattuto)
\end{itemize}

\subsubsection{Accuratezza}
L'accuratezza è una misura fondamentale perché: 
\begin{itemize}
	\item Misura la capacità del modello nel dare classificazioni affidabili su nuovi record
	\item Permette di selezionare l'istanza di modello che fornisce le migliori performance sui nuovi record
\end{itemize}
Consideriamo:

$D_T$ training set con $t$ record.

$D_{TS}$ test set con $v$ record.

$D = D_T \cup D_{TS},D_T \cap D_{TS} = \emptyset,m = t + v$\\
\textit{Un buon indicatore è la percentuale di record di test ($D_{TS}$) che sono classificati correttamente. }

Definiamo:
\begin{itemize}
	\item $y_i$ come il valore della variabile di classe dell'istanza $\underline{x_i} \in D_{TS}$
	\item $f(\underline{x_i})$ valore della classe predetto per l'istanza $\underline{x_i} \in D_{TS}$ dal modello di classificazione
\end{itemize} 
Consideriamo quindi la seguente funzione di Loss:
\[
L(y_i, f(\underline{x_i})) =  
\begin{cases}
	0 \qquad se &y_i = f(\underline{x_i}) \\
	1 \qquad se &y_i \ne f(\underline{x_i})
\end{cases}
\]
L'\textbf{Accuratezza} viene calcolata in questo modo:
\[ acc(D_{TS}) = 1 - \frac{1}{v} \sum_{i=1}^{v} L(y_i, f(\underline{x_i}))\]
In alcuni casi si preferisce l'\textbf{Errore} associato:
\[err(D_{TS}) = 1 - acc(D_{TS}) = \frac{1}{v} \sum_{i=1}^{v} L(y_i, f(\underline{x_i}))\]

\subsubsection{Speed}
Gli algoritmi di classificazione differiscono per:
\begin{itemize}
	\item Tempo di apprendimento
	\item Spazio di memoria occupato
\end{itemize}
Oggi però si dà meno importanza a queste caratteristiche in quanto vi sono diverse tecnologie e tecniche per ottimizzare i tempi e lo spazio di memoria costa sempre meno (economicamente parlando).

Un algoritmo che impiega molto tempo e/o una grande quantità di memoria per l'addestramento, può essere trainato dopo un campionamento del dataset originale. In questi casi accettiamo di non sfruttare tutte le informazioni disponibili invece di trainare un tipo di modello che riteniamo assicurerà buone performance.

\subsubsection{Robustezza}
\begin{defn}
	Un modello/algoritmo può essere \textbf{robusto} o meno rispetto a:
\begin{itemize}
	\item \textit{Outliers}: possono influenzare significativamente il modello
	\item \textit{Misssing data}: problema centrale (i dati di base sono sporchi e soggetti a deperimento)
	\item \textit{Variazione tra training set e test set}: si parte dal presupposto che i dati di training siano simili e utili per quelli di test, se questa affermazione decade non \`e pi\`u robusto l'algoritmo
\end{itemize}
\subsubsection{Scalabilità}
\end{defn}

\begin{defn}
	Si definisce \textbf{scalabilità} la capacità di apprendere da enormi quantità di dati. Questa proprietà è intresecamente connessa alla speed. 
\end{defn}
Es. le reti bayesiane scalano molto male. 


\subsubsection{Interpretabilità}
\begin{defn}
Quando la classificazione è orientata dall'\textbf{Interpretabilità} di un problema per essere risolto  non ci si può limitare ad assicurare alti valori di accuratezza, è importante estrarre regole semplici e chiare per l'esperto di dominio del caso di studio.
\end{defn}
L'interpretabilità è un problema enorme, perché noi umani non ragioniamo in termini quantitativi, ma qualitativi. Una spiegazione può essere la più dettagliata e precisa possibile ma se non intercetta il linguaggio e il contesto in cui vive il soggetto che voglio raggiungere, non viene raggiunto l'obiettivo di far interpretare il lavoro.

\subsubsection{Holdout}
\begin{defn}
	Si definisce \textbf{Holdout} la partizione del dataset $D$ in due sottinsiemi di training e test set attraverso un procedura di campionamento. 
\end{defn}

\textbf{Best practice} suggerisce 2/3 per il training set e 1/3 per il test set.

Se si hanno molti dati a disposizione non \`e necessario seguire questa proporzione si pu\`o aumentare sul training.\\

\begin{figure}[H]
	\centering
	\includegraphics[height=0.3 \linewidth]{classification/pict/holdout.png}
	\caption{procedura di Holdout}
\end{figure}
\textit{La stima dell'accuratezza dipende fortemente dalla particolare scelta di divisione tra training e test set.} Vi sono alcune tecniche per rendere più efficiente la divisione.

\begin{defn}
	Si definisce\textbf{Iterated Holdout} la tecnica che consiste nel ripetere iterativamente $r$ volte il metodo di holdout per cui apprendo e stimo l'accuratezza. 
\end{defn}
Ad ogni iterazione $r$ si estrae un campione random $D_{Tr}$ di $t$ record,  ottenendo: 
	
\[D_{TS_r} = D - D_{T_r}\]
	
Si ripete la procedura $r$ volte e l'accuratezza del classificatore è stimanata dalla media dei valori di accuratezza campionati $acc(D_{TS_r})$ calcolati su ogni test set $D_{TS_r}$:
	
\[ acc = \frac{1}{R} \sum_{r=1}^{R}acc(D_{TS_r})\]\\
	
Il numero di iterazioni $r$ può essere scelto tramite specifiche tecniche statistiche.

\textit{L'iterated Holdout è chiaramente un metodo più robusto, le sue performance hanno un bias più piccolo rispetto al normale holdout; tuttavia, non permette il controllo del numero di volte in cui un record è presente nel training e test set.} Non si ha la certezza di aver usato dei dati ottimi per la stima fatta. In alcuni casi si ricorre a uno schema più efficace, capace di ridurre l'impatto degli outlier. 

\begin{defn}
	Si definisce \textbf{Cross-validation} la tecnica di holdout che garantisce che ciascun record di un dataset $D$ sia incluso nel training set con lo stesso numero di volte ed esattamente una volta nel test set. 
\end{defn}
\textit{Il dataset $D$ viene partizionato in $K$ sottoinsiemi disgiunti detti \textbf{fold}, esaustivi e con circa lo stesso numero di record: }

\[D_1, D_2, ..., D_K\]

Eseguiamo $K$ iterazioni training-test. Alla k-esima iterazione avremo:

\[D_{T_k} = \{D_1, ...,D_{k-1},D_{k+1}, ...,D_K\} \qquad D_{TS_k} = D_k\]

L'insieme $D_{T_k}$ è usato per il training  mentre il $D_K$ viene usato per il test della k-esima iterazione, in pratica ho più training set e un test set.; l'accuratezza viene, quindi, stimata facendo la media delle $K$ iterazioni:

\[acc = \frac{1}{K} \sum_{k=1}^K acc(D_k)\]
Esistono diverse selezioni per il valore di $K$. I valori tipici sono $K = 3,5,10$ mentre un caso limite viene applicato quando i dati sono scarsi chiamato \textbf{Leave One Out Cross Validation} (\textbf{LOOCV}). \textit{LOOCV è una tecnica che assume che ogni record sia una partizione del dataset, così il valore di $K$ equivale al numero di record del dataset $D$.}

\begin{figure}[H]
	\centering
	\includegraphics[height=0.4 \linewidth]{classification/pict/cross_validation.png}
	\caption{Procedura di cross validation}
\end{figure}

Solitamente ci si chiede se ogni partizione del dataset contenga la \textit{stessa proporzione} di valori possibili della classe attributo. Quando le proporzioni sono fortemente sbilanciate si adottano specifiche tecniche di campionamento,es. \textbf{campionamento stratificato}. 
\`E buona norma fidarsi di più del \textbf{k-folds cross-validation} in quanto rispetto all'\textit{iterated holdout} si ha la certezza che i record non siano ripetuti.

\subsection{Comparing Classifiers (*)}
La comparazione tra due diversi \textit{classificatori} non è semplice, perché dipende dai differenti modelli utilizzati e dalla distribuzione dei dati, nonché dalla loro quantità. L'accuratezza ha diverso valore se ricavata da pochi record di dati. 
es. 
\begin{itemize}
	\item \textbf{Inducer A}: accuracy = 0.85 su un test set di 30 record
	\item \textbf{Inducer B}: accuracy = 0.75 su un test set di 5 000 record
\end{itemize}
Qual è il migliore?

\textit{L'accuratezza da sola non basta. Bisogna stimare l'intervallo di confidenza dell'accuratezza per i due inducer e testare l'importanza statistica delle deviazioni osservate.}
 
\subsubsection{Intervallo di confidenza}
Consideriamo il problema di predire il valore di una classe attributo da un test record come esperimento \textbf{binomiale}.

Dato un test set $D_N$ contenente $N$ record, consideriamo:
\begin{itemize}
	\item \textbf{X}: numero di record correttamente predetti dall'inducer
	\item \textbf{p}: vera, ma non la conosciuamo, accuratezza dell'inducer (probabilità di successo della distribuzione binomiale)
\end{itemize}
Modelliamo il problema con X distribuita secondo una binomiale con
\begin{itemize}
	\item media = $N \cdot p$
	\item varianza = $N \cdot p \cdot (1-p)$
\end{itemize}
L'oggetto del nostro studio è l'\textbf{accuratezza empirica} calcolata come:
 \[acc = \frac{X}{N} \] 
Che è distribuita secondo una binomiale con $\mu = p$ e $\sigma^2 = \frac{p \cdot (1-p)}{N}$

Sebbene la distribuzione binomiale possa essere usata per stimare l'intervallo di confidenza per l'accuratezza, se la dimensione del test set è sufficientemente grande, è buona norma approssimare la distribuzione con una \textbf{normale}. Così facendo l'intervallo di confidenza per l'\textbf{accuratezza empirica} diventa:

\[ P \Bigl( -Z_{1-\alpha/2} < \frac{acc - p}{\sqrt{p \cdot (1-p / N)}} < Z_{1 - \alpha/2} \Bigr) = 1 - \alpha \]\\

Rimodellando, la disuguaglianza ci porta al seguente intervallo di confidenza con confidenza $1-\alpha$ per \textbf{p}:

\[ \Biggl[ \frac{acc + \frac{Z_{1-\frac{a}{2}}^2 }{2 \cdot N} - Z_{1-\frac{a}{2}} \cdot \sqrt{\frac{acc}{N} - \frac{acc^2}{N} + \frac{Z_{1-\frac{a}{2}}^2}{4 \cdot N^2}}}{\bigl(1 + \frac{Z_{1-\frac{a}{2}^2}}{N}\bigr)}
, 
\frac{acc + \frac{Z_{1-\frac{a}{2}}^2 }{2 \cdot N} + Z_{1-\frac{a}{2}} \cdot \sqrt{\frac{acc}{N} - \frac{acc^2}{N} + \frac{Z_{1-\frac{a}{2}}^2}{4 \cdot N^2}}}{\bigl(1 + \frac{Z_{1-\frac{a}{2}^2}}{N}\bigr)} \Biggr] \]
\textbf{NB}: se prendessi diversi test set indipendenti l'uno dall'altro è ovvio che vengono diversi intervalli di confidenza. Però in questi casi fissato $\alpha$ posso stabilire statisticamente quali intervalli siano più significativi di altri. Più si  riduce $\alpha$ più aumenta $\beta$ ovvero il caso di errore in cui non viene rifiutata l'ipotesi nulla quando invece dovrei rifiutarla.

Dato un modello con acc=0.8 sul test set di $100$ record, fissiamo l'intervallo di confidenza per l'accuratezza a $0.95\%$. L'andamento dell'intervallo è il seguente:

\begin{figure}[H]
	\centering
	\includegraphics[height=0.5 \linewidth]{classification/pict/inter_confidence.png}
	\caption{Andamento dell'intervallo di confidenza}
\end{figure}

\subsubsection{Different test set}
Supponiamo il caso in cui sono presenti 2 modelli e vengono valutati su due test set diversi (è una cosa molto comune e che può accadere per numerevoli ragioni):
\begin{itemize}
	\item $M_1$ valutato su test set $D_1$ contenente $n_1$ record con tasso di errore $e_1$
	\item $M_2$ valutato su test set $D_2$ contenente $n_2$ record con tasso di errore $e_2$
\end{itemize}
I due test set devono essere assunti \textit{indipendenti}. Il nostro obiettivo è testare se la \textbf{differenza} tra i due errori è statisticamente significativa.
Assumendo che $n_1$ e $n_2$ siano sufficientemente grandi, allora i tassi di errore $e_1$ ed $e_2$ possono essere approssimati usando distribuzioni \textit{normali}. \\
La differenza osservata è denotata come: 
\[d = e_1 - e_2\]
Sono distribuiti secondo una \textbf{normale} con 
\[\mu = d_t \qquad \sigma^2 = \sigma^2_d\] 
La varianza di $d$ può essere stimata come segue: 
\[\sigma^2 \cong \hat{\sigma}_d^2 = \frac{e_1 \cdot (1-e_1)}{n_1} + \frac{e_2 \cdot (1-e_2)}{n_2}\]
L'intervallo di confidenza per la differenza $d_t$ è: 

\[ \bigl( d - z_{1-\alpha/2} \cdot \hat{\sigma}_d, d + z_{1-\alpha/2} \cdot \hat{\sigma}_d \bigr) \]
Ci possono essere sostanzialmente 3 casi:
\begin{enumerate}
	\item \[0 \in \bigl( d - z_{1-\alpha/2} \cdot \hat{\sigma}_d, d + z_{1-\alpha/2} \cdot \hat{\sigma}_d \bigr)\] 
	In questo caso si conclude che la differenza osservata \textbf{non è statisticamente significativa} ad un livello $\alpha$, i modelli non sono significativamente differenti 
	\item \[d + z_{1-\alpha/2} \cdot \hat{\sigma}_d < 0\] 
	Se il limite superiore della confidenza è negativo allora il modello \textbf{$M_1$ è migliore del modello $M_2$} a un livello $\alpha/2$
	\item \[d - z_{1-\alpha/2} \cdot \hat{\sigma}_d > 0\] 
	Se il limite inferiore della confidenza è positivo allora il modello \textbf{$M_2$ è migliore del modello $M_1$} a un livello $\alpha/2$
\end{enumerate}

\`E buona norma fare il test di ipotesi \textbf{solo} se prima mi faccio la domanda se siano o meno differenti, non bisogna  condurre test a casaccio o a tappeto perché altrimenti si rischia di raggiungere dei risultati assurdi. Bisogna porsi la domanda e poi eseguire tutta la procedura sulla coppia di modelli, in quanto vi è un \textbf{problema dei confronti multipli} devo adattarli tutti allo stesso livello di $\alpha$ altrimenti le differenze non sono confrontabili.

\subsubsection{Same test set}
Nel caso in cui sia possibile utilizzare lo \textbf{stesso test set} si può svolgere un test più \textbf{potente}; ovvero, si può valutare il caso di errore di secondo tipo: si afferma che la differenza tra i due classificatori non è significativa quando in realtà lo è ($\beta$). 

Si confrontano $M_1$ ed $M_2$ usando il \textit{k-fold cross validation}.
Ho il dataset $D$ partizionato in $K$ subset disgiunti con circa lo stesso numero di record. 
\[D_1, D_2, ..., D_K\]
Si applicano le tecniche di classificazione per costruire i modelli $M_1$ e $M_2$ dalle $k-1$ partizioni e si testano con la partizione rimanente. Il passaggio deve essere ripetuto $K$ volte, ogni volta usando una differente partizione per il test set. Si ottiene:
\begin{itemize}
	\item $M_{1k}$ \textbf{inducer} per il modello $M_1$ ottenuto alla k-esima iterazione con $e_{1k}$ errore
	\item $M_{2k}$ \textbf{inducer} per il modello $M_2$ ottenuto alla k-esima iterazione con $e_{2k}$ errore
\end{itemize}
La differenza tra gli errori durante la k-esima iterazione è:
\[d_k = e_{1k} - e_{2k}\]
Per K sufficientemente grande, allora $d_k$ è distribuito \textit{normalmente} con: 
\[\mu = d_t^{cv} \quad \sigma = \sigma^{cv}\]
Dove la varianza osservata è stimata utilizzando la seguente formula: 

\[ \hat{\sigma}^2_{d^{cv}} = \frac{\sum_{k=1}^{K}(d_k - \bar{d})^2}{K \cdot (K-1)} \qquad \bar{d} = \frac{1}{K} \sum_{k=1}^K d_k \]
Usiamo una distribuzione T di Student per calcolare l'intervallo di confidenza per il valore della vera media $d_t^{cv}$

\[\biggl( \hat{d} - t_{1-\frac{\alpha}{2}}^{K-1} \cdot \hat{\sigma}_{d^{cv}}
,
\hat{d} + t_{1-\frac{\alpha}{2}}^{K-1} \cdot \hat{\sigma}_{d^{cv}} \biggr)\]
Dove 
\[t_{1-\frac{\alpha}{2}}^{K-1}\]
\`E ottenuta dalla tavola di probabilità, il quantile associato alla confidenza $1-\alpha$ e $K-1$ gradi di libertà.\\
Ovviamente valgono \textbf{le stesse considerazioni} fatte precedentemente sull'intervallo di confidenza per la differenza tra due classificatori: se l'intervallo di confidenza contiene il valore 0, concludiamo che la differenza osservata \textit{non è statisticamente significativa} al livello $\alpha$ (confidenza $1-\alpha$). 

\subsection{Class Imbalance Problem (*)}
Consideriamo di nuovo il dataset dei Churner. In particolare è noto che il $14.5\%$ sono dei churner. Un modello di classificazione che etichetta i record come non churner avrà per forza un'alta accuracy in quanto è la classe più frequente nel dataset. 

\begin{defn}
	Si definisce la \textbf{ZeroR Rule}	il modello che risponde sempre con la classe più frequente, risultando, quindi, il modello più inutile.	
\end{defn}
La misura di \textit{accuracy} tratta classi equamente importanti, non è adatta ad analizzzare \textbf{dataset sbilanciati},  dove la classe più rara è considerata \textit{più interessante} di quella più frequente. Per la classificazione binaria si distingue:
\begin{itemize}
	\item La \textbf{classe positiva} ovvero la classe più rara (o meno frequente)
	\item La \textbf{classe negativa} ovvero la classe più frequente
\end{itemize}
Consideriamo la \textbf{matrice di confusione}.
\begin{figure}[H]
	\centering
	\includegraphics[height=0.25 \linewidth]{classification/pict/matrconf.png}
	\caption{Matrice di confusione binaria}
\end{figure}
Dalla matrice calcoliamo i seguenti indicatori:
\begin{itemize}
	\item 		\textbf{TNR, specificità} (\textit{tasso veri negativi}):  frazione di record negativi correttamente predetti dal modello  \[TNR = \frac{TN}{TN + FP}\]
	\item	\textbf{TPR, sensitività} (\textit{tasso veri positivi}): frazione dei record positivi correttamente predetti dal modello \[TPR = \frac{TP}{TP + FN}\]
	\item \textbf{FPR, tasso falsi positivi}: frazione dei record negativi predetti come classe positiva dal modello \[FPR = \frac{FP}{TN + FP}\]
	\item \textbf{FNR, tasso falsi negativi}: frazione dei record positivi predetti come classe negativa dal modello \[TNR = \frac{FN}{TP + FN}\]
\end{itemize}
Le metriche che si calcolano nella ricerca di una classe più importante delle altre sono la \textbf{Precision} e il \textbf{Recall}.
\begin{defn}
	La \textbf{Precision} determina la frazione di record che effettivamente si rivela essere positiva nel gruppo che il classificatore definisce come classe positiva. 
	\[p = \frac{TP}{TP+FP}\]
\end{defn}
Si prende il punto di vista del classificatore. Valuto quanti di quelli positivi ha valutato come effettivamente positivi. Più è \textit{alta}  la precisione più è \textit{basso} il numero di falsi positivi commessi.

\begin{defn}
	L'indice \textbf{Recall} misura la frazione di record positivi correttamente predetti dal modello di classificazione; in particolare, è definito come:
	\[r = \frac{TP}{TP+FN}\]
\end{defn}
In questo caso si prende il punto di vista della realtà. Al denominatore si hanno quelli effettivamente positivi, al numeratore ho quelli che il classificatore ritiene positivi. Un \textit{alto} Recall signfica \textit{pochi} record positivi \textit{erroneamente} classificati come classe negativa. Infatti, il Recall è equivalente al $TPR$.

Vediamo un esempio sul problema del churn:
\begin{figure}[H]
	\centering
	\includegraphics[width=\linewidth]
	{classification/pict/esPrecisionRecall_merge.png}
	\caption{Esempio calcolo precision e recall}
\end{figure}
Vi è un \textit{legame molto stretto tra questi due indici}, si può avere un recall molto alto ma allora probabilmente avrò una precision bassa. Oppure si può puntare ad avere una precisione alta ma in quei casi probabilmente si avrà una recall bassa.

\textit{Queste due misure per convenzione sono calcolate sulla classe positiva (quella meno frequente) non \`e per\`o detto che non vadano utilizzate su quella negativa. Inoltre \textit{non} \`e sempre detto che queste quantit\`a siano sempre calcolabili (possibile divisione $0/0$).}

Siccome queste due misure sono legate (pensa alla problema della coperta corta). Si utilizza una misura che le riassume.
\begin{defn}
	Si definisce \textbf{$F_1$ mesure}  la media armonica tra \textbf{Recall} e \textbf{Precision}. 
	\[ F_1 = \frac{2 \cdot r \cdot p}{r + p}\]
\end{defn}	
Per come è definita un alto valore di $F_1$ implica alti valori di recall e precision.
Esiste una generalizzazione sotto il nome di \textbf{$F_\beta$ mesure} per esaminare il tradeoff tra Precision e Recall:
	\[ F_\beta = \frac{(\beta^2 + 1) \cdot r \cdot p}{r + \beta^2 \cdot p} \]
\begin{itemize}
	\item $F_\beta$ con $\beta = 0$ è la Precision
	\item $F_\beta$ con $\beta = \infty$ è la Recall
\end{itemize}


\subsection{Counting the cost (*)}

\subsubsection{Matrice di costo}
Se pensiamo al nostro problema del churner, l'azienda ha interesse ad evitare quello che si prevede, ovvero che il cliente se ne vada, in sostanza si cerca di prevedere chi vuole andarsene e si cerca di evitare che questa previsione sia effettiva applicando politiche di dissuasione. L'azienda può spendere un tot budget per invertire questo fenomeno, bisogna fare il meglio per identificare i possibili churner e attuare \textbf{solo} su di loro le politiche di dissuasione (se lo facessi per tutti i clienti non avrebbe senso per quanto riguarda i costi).
\\Bisogna convincere il proprio interlocutore che il modello sviluppato sia migliore rispetto al modello che storicamente hanno utilizzato. La misura più utilizzata per questo è l'accuracy legata alla matrice di confusione. In aggiunta, però, bisogna associarci la \textbf{matrice di costo}. 

\begin{defn}
	La \textbf{Matrice di Costo} stabilisce in base a falsi positivi e negativi quanto costo (in soldi/lavoro) sostiene l'azienda per classificare un certo soggetto (es. come churner o non churner). 
\end{defn}
	Il costo è così calcolato:
\begin{figure}[H]
	\centering
	\includegraphics[height=0.55 \linewidth]{classification/pict/matr_costo.png}
	\caption{Legame tra matrice di confusione (sn) e matrice di costo (dx)}
\end{figure}
\[Cost = C_{--} \cdot TN + C_{-+} \cdot FP + C_{+-} \cdot FN + C_{++} \cdot TP\]
\textbf{NB} se matrice di costo è simmetrica allora il costo corrisponde all'accuratezza; per comprenderlo, vediamo il seguente esempio. Confrontiamo le performance di accuracy e costo tra lo \textit{Standard Mailout Procedure} (\textbf{SMP}), procedura già utilizzata per dissuadere i clienti, con il modello nuovo di classificazione \textbf{M} da noi costruito:

\begin{figure}[H]
	\centering
	\includegraphics[width=\linewidth]{classification/pict/esConfMatr.png}
	\caption{Matrici di confusione: a sinistra SMP, a destra il modello M}
\end{figure}
Ora, però, bisogna associarci la \textbf{matrice di costo} per le rilevazioni e ne risulta:
\begin{figure}[H]
	\centering
	\includegraphics[height=0.2 \linewidth]{classification/pict/esMatrCosto.png}
	\caption{Matrice di costo churner}
\end{figure}
Come si può notar,e nonostante il nostro modello abbia una misura di \textbf{accuratezza} migliore, ha un \textbf{costo} (in termini di soldi) superiore rispetto al modello storicamente utilizzato dall'azienda. Pertanto, in questo caso, si preferirà la vecchia procedura SMP piuttosto che la nostra soluzione.
\`E importante notare però che ci sono situazioni nelle quali perdere un cliente è più grave che perderne un altro. Devo capire se sto utilizzando la matrice dei costi vera, oppure, se è solo rappresentativa. Se essa non è certa posso ragionare in un altro modo.

\subsubsection{Cumulative Gains} 
Supponiamo che ci sia una popolazione di 1.334 clienti di cui 500 sono possibili churners. Per esperienza accumulata sappiamo che il $15\%$ dei clienti è un churner, quindi:
\\$1334*0.15 = 200$ churner sul totale;
\\$500*0.15 = 75$ churner sui potenziali churners.
\\\\
Se si considerasse positiva la classe dei churner e negativa quella dei non churner e si provasse un campionamento casuale su di essi si avrebbe questo risultato:
\begin{figure}[H]
	\centering
	\includegraphics[height=0.1 \linewidth]{classification/pict/esChurnerRandom.png}
\end{figure}

Il modello di classificazione sviluppato, se applicato, identifica sui potenziali churner il $60\%$ di chi effettivamente se ne va: 

$200*0.6 = 120$ churners corretamente identificati.
\begin{figure}[H]
	\centering
	\includegraphics[height=0.1 \linewidth]{classification/pict/esChurnerModel.png}
\end{figure}
Pertanto possiamo affermare che il nostro modello sia molto meglio della \textit{zeroRule}, ovvero il campionamento casuale. 

\begin{defn}
	Si definisce \textbf{Lift Factor} il rapporto tra il modello $M$ rispetto al campionamento casuale:
	
	\[Lift = \frac{performace(M)}{performance(zeroRule)}\]
\end{defn}
Questo coefficiente è usato per calcolare l'incremento di performance tra due modelli.
Nel nostro caso: $0.6/0.375 = 1.6$\\

\textbf{NB}: si considera che il nostro classificatore è bravo ad identificare i casi semplici ma rischia di sbagliare di molto in quelli complessi. Man mano che viene forzato a rispondere lui tenderà a rispondere in modo sempre più random.

Si può ora sfruttare il \textit{Lift factor} per comprendere il livello di \textbf{profittabilità} una volta che i costi coinvolti sono noti. In questo caso bisogna avere un sottoinsieme di customers che hanno \textit{alta proporzione} di record positivi, \textit{maggiore} rispetto al dataset di partenza.

\begin{enumerate}
	\item \textbf{Ordinare} gli output con probabilità di riconoscimento corretto \textbf{prob(y)} in ordine decrescente
	\item \textbf{Estrarre} un primo sottoinsieme (partendo dall'alto) e calcolare il \textit{lift factor}, per forza di cose sarà alto
	\item \textbf{Considerare} allora un sottoinsieme più grande (sempre partendo dal primo valore) e calcolare il \textit{lift factor}, ci si aspetta un valore minore rispetto al precedente
	\item \textbf{Ripetere} il punto 3 finché si includono tutti i record
\end{enumerate} 

Nell'esempio in figura due passaggi del procedimento 
\begin{figure}[H]
	\centering
	\includegraphics[height=0.6 \linewidth]{classification/pict/liftFactor.png}
	\caption{Calcolo del \textbf{lift factor} primo sottoinsieme (sinistra) e secondo sottoinsieme (destra)}
\end{figure}

\begin{defn}
	I valori di Lift factor così calcolati vanno a formare la cosiddetta curva dei \textbf{Cumulative Gains}.
\end{defn}
\begin{figure}[H]
	\centering
	\includegraphics[height=0.6 \linewidth]{classification/pict/cumulative_gains.png}
	\caption{Cumulative gains evidenziando un punto di alto valore aggiunto}
\end{figure}

\textbf{NB}: La retta {\color{ao(english)}verde} calcola la cumulative gain per un modello che risponde in maniera \textit{casuale}, sull'asse x è la \textit{recall}, i positivi riconosciuti correttamente (ripetendo il test ad ogni step scelto)

\subsubsection{Lift Chart}
La curva \textit{Cumulative Gains} può essere mappata direttamente nella \textbf{Lift Chart}.

\begin{figure}[H]
	\centering
	\includegraphics[height=0.45 \linewidth]{classification/pict/lift_chart.png}
	\caption{Da cumulative gain a lift chart}
\end{figure}

Queste valutazioni permettono di comprendere \textit{quando} il classificatore perde efficacia. Naturalmente più il campione è piccolo meglio funziona, in quanto si usa record con \textbf{prob(y)} di riconoscimento corretto molto alta, bisogna comprendere però il punto di massima accuratezza in rapporto alla percentuale di istanze.

\subsubsection{Curva ROC}

\begin{defn}
	\`E chiamata \textbf{ROC} e sta per Receiver Operating Characteristic curve la tecnica grafica per la valutazione di modelli di classificazione.
\end{defn} 
Assomiglia molto alla cumulative gains ma sugli assi sono presenti:
\begin{itemize}
	\item \textbf{asse x:} la percentuale dei record falsi positivi sopportabili (\textbf{FPR})
	\item \textbf{asse y:} la percentuale dei record veri positivi (\textbf{TPR})
\end{itemize}
Entrambi i valori sono espressi in \textit{percentuale sul totale}. 
\begin{figure}[H]
	\centering
	\includegraphics[height=0.6 \linewidth]{classification/pict/roc.png}
	\caption{Curva ROC}
\end{figure}
\textit{Nell'esempio in figura vengono utilizzati particolari sottoinsiemi di dati, questa dipendenza può essere ridotta applicando la Cross-validation. La curva ROC rappresenta la performance di un classificatore senza guardare la distribuzione della classe o il costo dell'errore; serve, quindi, a confrontare diversi classificatori per cercare di comprendere dove un classificatore \`e pi\`u o meno efficace}; un esempio è il seguente:

\begin{figure}[H]
	\centering
	\includegraphics[height=0.6 \linewidth]{classification/pict/roc_confronto.png}
	\caption{Confronto ROC tra due modelli}
\end{figure}

Un modello è \textbf{preferibile} ad un altro se è disposto a sopportare un certo numero di falsi positivi.


\subsection{Feature Selection (*)}
Quasi sempre non ha senso utilizzare tutti gli attributi per generare il modello che si vuole sviluppare; Si procede, quindi,  con la selezione delle feature.
La \textbf{feature selection} è un compito importantissimo ed è fortemente legato alla risoluzione del problema. Analizzare le relazioni tra le variabili in modo \textit{preliminare} potrebbe aiutare di molto nella ricerca della soluzione migliore al problema. 
\begin{defn}
La \textbf{feature selection} è il processo che tenta di scoprire quali attributi sono:
\begin{itemize}
	\item \textit{Ridondanti} ovvero quelli dei quali l'informazione è già contenuta in altri attributi, la rilevanza va sempre valutata in base ai dati a disposizione
	\item \textit{Irrilevanti} contengono informazioni non utili per risolvere il problema di data mining.
\end{itemize}
\end{defn}

Vi sono diversi \textbf{approcci} per rilevare questi attributi:
\begin{itemize}
	\item \textit{Brute-force:} si provano tutti i modelli per ogni combinazione di parametri usati nel modello di Classificazione. La computazione di questi casi è troppo grossa, vi sono: \[\sum_{n=1}^{10}\binom{10}{n}\] possibili combinazioni
	\item \textit{Embedded:} gli attributi sono scelti in base alla capacità del modello della classificazione
	\item \textit{Filter}: gli attributi sono selezionati prima del classificatore in base a quelli che si ritengono più rilevanti e meno rilevanti (analisi di funzione obiettivo)
	\item \textit{Wrapper}: gli attributi si scelgono in base al modello di classificazione scelto, quelli che sono rilevanti per un modello non lo sono per un altro (dipende dall'ipotesi che ho fatto)
\end{itemize}
Politica Filter \& Wrapper a confronto
\begin{figure}[H]
	\centering
	\includegraphics[height=0.5 \linewidth]{classification/pict/filter_wrapper.png}
	\caption{Filter - Wrapper, due differenti approcci per feature selection}
\end{figure}
\textit{Non va bene fare feature selection su tutti i dati e poi trainare solo su un sottoinsieme!! altrimenti si hanno risultati non comparabili}

\subsubsection{Filter} 
Nel caso di attributi \textbf{Uni-variati}: 
\begin{enumerate}
	\item Si sceglie una \textit{misura di associazione} tra gli attributi candidato e quello di classe
	\item Si \textit{ordinano} gli attributi in base alla misura di associazione
	\item Si \textit{selezionano} le prime $R$ migliori posizioni come attributi di input per il classificatore
\end{enumerate}
Solitamente seguendo questa procedura si identificano correttamente gli \textit{attributi irrilevanti}; non è detto, però, che funzioni bene nella ricerca di \textit{attributi ridondanti}.
Nel caso di attributi \textbf{Multi-variati}:
\begin{itemize}
	\item Si identificano congiuntamente \textit{attributi rilevanti} e \textit{irrilevanti}
	\item Un buon sottoinsieme di attributi deve contenere attributi fortemente associati con l'attributo di classe ma essere incorrelati tra di loro
\end{itemize}
Le tecniche più utilizzate sono:
\begin{figure}[H]
	\centering
	\includegraphics[height=0.35 \linewidth]{classification/pict/feature_tecniques.png}
	\caption{Tecniche uni-multi variate per filtrare le features}
\end{figure}
\begin{figure}[H]
	\centering
	\includegraphics[width=\linewidth] {classification/pict/filter_uni_multi.png}
	\caption{Vantaggi e svantaggi tra tecniche uni-variate e multi-variate}
\end{figure}
\noindent
I \textbf{vantaggi} per la feature selection sono:
\begin{itemize}
	\item Riduzione del costo di collezione di dati
	\item Riduzione dei tempi di inferenza del classificatore relativo all'attributo di classe
	\item Classificatore pi\`u interpretabile
	\item Aumento dell'accuratezza
\end{itemize}
Le \textbf{motivazioni} per cui questo approccio viene utilizzato sono:
\begin{itemize}
	\item Evitare l'overfitting
	\item Sviluppo di un miglior cost-effective classificatore
	\item Migliorare la comprensione del processo di generazione dati
\end{itemize}

Flowchart da seguire:
\begin{figure}[H]
	\centering
	\includegraphics[height=0.45 \linewidth]{classification/pict/feature_flowchart.png}
	\caption{flowchart per la feature selection}
\end{figure}

È spesso possibile creare, dagli attributi originali, un nuovo insieme di attributi che cattura informazioni rilevanti nel dataset in modo più efficace; inoltre, il numero di nuovi attributi può essere più piccolo rispetto al numero di attributi originali.\\
Le metodologie per la creazione di nuovi attributi (features) sono:
\begin{itemize}
	\item \textbf{Feature extraction}: una ampia gamma di modelli possono essere applicati in domini specifici come per classificazione di immagini/segnali
	\item \textbf{Mappare dati in un uovo spazio}: vedere gli attributi in modo normalizzato o secondo un'altra scala più facilmente trattabile (es. SVM) può migliorare di molto le performance
	\item \textbf{Feature construction}: generare feature nuove più comode per un certo modello di classificatore rispetto a quelle originali attraverso trasformazioni, es. logaritmi di somme ecc.. (non \`e detto che i dati originali siano i pi\`u utili per la classificazione)
\end{itemize}
\subsubsection{Regularization}
Alcuni modelli di Machine Learning ammettono iper-parametri. \`E dimostrato che una buona scelta dell'architettura di una rete neurale  permette  di approssimare qualsiasi problema. Non è però esente da overfitting, anzi, visto che non è sempre chiaro il funzionamento, è più difficile valutare quando sia presente. Si cerca di scegliere quella allocazione di parametri che riduce il più possibile l'errore quadratico del modello, quindi l'iperparametro $\lambda$ detto di regolarizzazione:

\[E(\textbf{w},\lambda) = \frac{1}{m} \sum_{i=1}^{m}(y_i - \hat{y}_i)^2 + \frac{\lambda}{2} \sum_{j=1}{K} w_i^2 \]

$K$: numero di parametri liberi (weights + thresholds) della rete neurale

$\lambda$: parametro di regolamentazione\\
La prima parte della formula riguarda il fitting della funzione, la seconda parte \`e la flessibilit\`a alla quale posso accedere.\\

\textit{Non si deve usare il training set per ottimizzare il parametro di regolarizzazione $\lambda$. Si andrebbe ad overfittare il modello. Se viene utilizzato il test set per ottimizzare il parametro lambda allora si è bruciato il dataset e devo rifare il modello, l'obiettivo è sempre quello di riconoscere dati mai visti, per questo si adottano schemi di divisione de dataset diversi.}

\subsubsection{Schema division}
Lo schema di divisione del dataset ottimale potrebbe essere questo:
\begin{figure}[H]
	\centering
	\includegraphics[height=0.3 \linewidth]{classification/pict/schema_dataset.png}
	\caption{Diversi schemi di divisione del dataset}
\end{figure}

\begin{itemize}
	\item Il train dataset viene utilizzato per trainare il modello (sui parametri w)
	\item Il parametro $\lambda$ viene ottimizzato usando il validation set
	\item Il test set viene usato per fornire una stima delle performance senza bias
	\item Può essere fatto tramite holdout, iterated holdout e cross-validation
\end{itemize}
Quando viene usato l'approccio \textbf{Filter} per la feature selection solitamente si usa uno schema \textbf{Train/Test}
\begin{itemize}
	\item Rilevanza e/o ridondanza sono stimate usando il Train set
	\item Dopo aver selezionato gli attributi il Train set viene usato per allenare il classificatore
	\item Le stime delle performance sono ottenute applicando il classificatore al Test set
\end{itemize}
Quando viene applicato l'approccio \textbf{Wrapper} per la feature selection solitamente si usa uno schema \textbf{Train/Validation/Test}
\begin{itemize}
	\item Il Validation set viene usato per ottimizzare le performance quando vengono usati diversi attributi per il modello (i wrapper)
	\item Selezionare sottoinsiemi ottimali di attributi è lo stesso che selezionare il valore ottimo del parametro regolarizzato $\lambda$
	\item Una volta selezionati gli attributi, il Train set e il Validation set vengono uniti e usati per addestrare il classificatore (tipicamente lo stesso tipo di classificatore usato per la feature selection)
	\item Le stime delle performance sono ottenute applicando il classificatore al Test set
\end{itemize}

\subsection{Classificazione NON binaria}
Nei problemi reali la classificazione avviene su più valori non solo binari.
\begin{defn}
	Un problema di classificazione \textbf{Non binaria} è un problema di classificazione che si divide in:
	\begin{itemize}
		\item \textbf{Multi-classe}: esattamente una classe si realizza
		\item \textbf{Multi-etichetta}: più di una classe può verificarsi
	\end{itemize}
\end{defn}
Vi possono essere anche problemi non di classe ma di \textbf{ranking} in cui i valori assunti dalla variabile di classe sono ordinati.
Per la risoluzione di questi problemi solitamente si adotta una logica \textbf{One-Vs-All}.

\subsubsection{One-Vs-All}
L'idea è quella di trasformare un problema multi-classe in tanti attributi di classe binari. 
In questa modalità si verifica se il set corrente verifica o meno ciascuna delle caratteristiche da valutare. 

\textit{Bisogna ricordare che non sempre binarizzare è necessario, in particolare per quanto riguarda i naive bayes non serve, invece per un decision tree sì.} 
\\Nel \textbf{One-vs-all} si crea un numero di \textit{classificatori binari} diverso in base al numero di modalità dell'attributo di classe; bisogna, inoltre, normalizzare poi gli output dei classificatori. 
Nel modello \textit{Multi-class} bisogna raccogliere i valori risultanti dai classificatori e fissare un threshold sopra il quale considero signficiativo il risultato (potenzialmente pi\`u di una classe supera il threshold).
\begin{figure}[H]
	\centering
	\includegraphics[height=0.4 \linewidth]{classification/pict/es_no_binary_classification.png}
	\caption{Esempio di classificatori binari}
\end{figure} 

\clearpage
\restylefloat{table,figure}
\pagestyle{fancy}
\cfoot{\thepage}
\renewcommand{\footrulewidth}{0.25pt}


 
\section{Clustering}
\subsection{Introduzione}

L'analisi di cluster è usata per risolvere moltissimi problemi pratici. In particolare l'\textbf{analisi di cluster tratta due diversi scopi generali:}

\begin{itemize}
	\item \textbf{Comprensione}: Le classi, o gruppi di oggetti che condividono caretteristiche, giocano un ruolo importante nella comprensione del mondo. Questo succede in biologia, in informatica e in economia.
	\item \textbf{Utilità}: Capacità di risassumere determinate carateristiche di un oggetto con le caratteristiche del cluster a cui appartiene. L'obiettivo è \textit{trovare i prototitpi con le proprietà più rappresentative dei cluster}.
\end{itemize}

Possiamo dare ora una definizione un po' più completa di analisi di clustering.
\begin{defn}
	La \textbf{cluster analysis} raggruppa i dati basandosi sulle informazioni trovate nei dati che descrivono gli oggetti e le loro relazioni.
\end{defn}

Gli obiettivi sono ora semplici da ridefinire:
\begin{itemize}
	\item Gli oggetti all'interno di un gruppo devono essere simili gli uni con gli altri, allo stesso tempo diversi (o incorrelati) con gli oggetti di altri gruppi.
	\item La più grande \textit{similarità} entro un gruppo deve corrispondere ad una grande \textit{differenza} tra i gruppi. 
\end{itemize}

\underline{Problema}: Come faccio a stabilire quando  e quanto degli oggetti sono simili? Non vi è un metodo per capirlo, tendenzialmente si utilizza una soluzione intermedia rispetto alle altre. 

La definizione di cluster è \textbf{intrinsicamente imprecisa},una migliore definizione dipende infatti dalla natura dei dati e dai risultati desiderati. 

\subsubsection{Tipi di clustering}

Formare un insieme di cluster è chiamato in gergo tecnico \textit{clustering.} Ci sono diversi tipi di analisi di clustering.

\begin{itemize}
	\item Partitional vs Hierarchical
	\item Exclusive vs  Overlapping vs Fuzzy
	\item Complete vs Partial.
\end{itemize}

Diamo ora una rapida definizione di tutte:

\begin{defn}
	Un clustering si dice \textbf{partizionale} se vi è una divisione del dataset in insiemi non sovrapposti tali che un elemento appartiene ad un solo insieme.
\end{defn}

\begin{defn}
	Un clustering si dice \textbf{gerarchico} se ogni cluster può essere a sua volta suddiviso in sotto cluster, in questo caso il clustering è un insieme di cluster che sono organizzati come un albero.
\end{defn}

\begin{defn}
	Un clustering si dice \textbf{esclusivo} se ogni oggetto è assegnato ad un singolo cluster.
\end{defn}

\begin{defn}
	Un clustering si dice \textbf{sovrapponibile} se ogni oggetto può essere assegnato a più di un cluster.
\end{defn}

\begin{defn}
	Un clustering si dice \textbf{fuzzy} se ogni oggetto può essere assegnato a più di un cluster contemporaneamente con un valore che tiene conto del peso che ha l'oggetto rispetto all'appartenenza ad un singolo cluster, la somma dei pesi deve essere necessariamente 1.
\end{defn}
\begin{figure}[H]
	\centering
	\includegraphics[height=0.3 \linewidth]{clustering/pict/fuzzy.png}
	\caption{Custering con modello fuzzy}
\end{figure}
\begin{defn}
	Un clustering si dice \textbf{completo} se ogni oggetto è assegnato ad un cluster (non ci sono oggetti liberi).
\end{defn}

\begin{defn}
	Un clustering si dice \textbf{parziale} se esiste almeno un oggetto che non è assegnato a nessun cluster, si usa perché potrebbero esserci degli outlier ed inserirli all'interno di un cluster potrebbe peggiorare in modo significativo la rappresentazione di un cluster.
\end{defn}

\subsubsection{Differenti nozioni di cluster}
I cluster naturali sono cluster che si dicano esistano per davvero anche se questo  è molto difficile che ciò accada. Per visualizzare le differenze tra i tipi diversi di dati sfruttiamo dati come punti a due dimensioni.
\begin{itemize}
	\item \textbf{Well separated Cluster:} dato un cluster, ogni oggetto è più vicino ad ogni oggetto del cluster a cui appartiene piuttosto che a qualsiasi altro oggetto di ogni altro cluster. Un cluster così ben formato permette di avere separazioni molto nette, questo tipo di cosa succede però molto raramente in realtà.
	\item \textbf{Prototpe-based cluster} dato un cluster, ogni oggetto di quel cluster è più vicino al prototipo che definisce il cluster rispetto ad ogni prototipo di un altro cluster. Il prototipo \`e solitamente il \textit{centroide} del cluster. Il \textit{prototipo} di un cluster corrisponde all'individuo meglio rappresentato dal cluster (può essere anche fittizio). Cluster fatti in questo modo tendon ad essere globulari.
	\item \textbf{Density-based cluster} un cluster è una regione densa di oggetti che sono circostritti da una regione di bassa densità. Questi sono usati quando i cluster sono irregolari o intermittenti oppure quando c'è una grande presenza di rumore o di outlier.
	\item \textbf{Graph-based cluster} se i dati sono rappresentati da grafi, i nodi rappresentano  oggetti e i collegamenti connettono gli oggettti. Allora ogni cluster è una componente connessa. La connessione può anche essere pesata e scelta in base ad una certa soglia. 
	Questi cluster sono molto utilizzati in quanto c'è un sacco di ricerca già fatta.
\end{itemize}

\subsubsection{Componenti di un'analisi di clustering}

Per prima cosa  avviene la \textit{feature selection} che assicura la trattenuta degli attributi del dataset degni di significato. Successivamente avviene la fase di  \textit{feature extraction} che serve a produrre feature che potrebbero andare meglio per scoprire la struttura dei dati, questa pratica potrebbe tuttavia generare features di difficile comprensione.
Bisognerebbe usare come feature ideali quelle che permettono di distinguere i pattern degli elementi che appartengono ai diversi cluster, immuni al rumore e facili da interpretare.

Il secondo passo è quello di \textit{determinare la misura di prossimità e costruire la funzione di merito.} Una volta che abbiamo determinato una misura di prossimità il problema di clustering si traduce in un problema di ottimizzazione con una specifica funzione.

Bisogna ricordare sempre che diversi algoritmi di dati permettono di avere conclusioni anche totalmente diverse, questo è il motivo per cui in principio non bisogna prediligere alcun algoritmo ma confrontare i risultati ottenuti e trarne conclusioni.


\subsection{ Proximity}

La proximity è uno strumento fondamentale per valutare il funzionamento di un algoritmo di clustering. Questa nfluenza quindi in modo pesante la nostra soluzione di un nostro problema di clustering. Bisogna fare una scelta della misura con cui approssimiamo quanto sono simili gli elementi appartenenti allo stesso cluster.
\subsubsection{Introduzione}
\textit{La analisi dei cluster affonda le sue radici nel concetto di cosa sia simile e cosa sia dissimile,} quando cerchiamo di esprimere questo concetto  in termini formali diventa abbastanza difficile. Questo avviene perché la similitudine dipende fortemente dal contesto analizzato .

In generale la similarità è nulla se i due oggetti sono totalmente differenti sotto la caratteristica che stiamo valutando ed è uguale a 1 se sono completamente uguali, è comune però trovare misure di similitudine che hanno come valori:.  $[0,\inf]$. Useremo il termine \textbf{proximity} per indicare sia la similarità che la dissimilarità. 

\begin{figure}[H]
	\centering
	\includegraphics[height=0.45 \linewidth]{clustering/pict/simil_diss.png}
	\caption{vantaggi e svantaggi tra tecniche uni-variate e multi-variate}
\end{figure}

Non c'è ortogonalità tra la scelta della misura e l'esito che otterrò. Nasce per questo motivo l'esigenza di poter passare da una misura all'altra, in particolare per trasformare una misura di similarità dall'intervallo $[0,\inf]$ all'intervallo $[0,1]$ si opera la seguente trasformazione:

\[s' = \frac{s - \min{s}}{\max{s} - \min{s}} \qquad d' = \frac{d - \min{d}}{\max{d} - \min{d}} \] 

Ci sono diversi problemi che nascono quando cambiamo l'intervallo in cui si trova il valore. Per farlo devo usare una trasformazione \textit{non-lineare}. Posso usare però una cosa del genere:

\[ d' = \frac{d}{1+d}\]

Con questa trasformazione grandi valori della dissimilarità $d$ vengono compressi in valori vicini a 1. Il fatto di distorcere o meno le distanze dipende dal compito che voglio svolgere.

Ci sono diversi problemi che nascono quando trasformiamo una similarità in dissimilarità e viceversa. Per farlo devo usare una trasformazione \textit{non-lineare}. Posso usare però una cosa del genere:

\[ se \quad s,d = [0, 1]  \quad allora  \quad s =  1-d\]

In generale si può usare qualsiasi funzione monotona decrescente per trasformare la similarità in dissimilarità.

\begin{defn}
	Si definisce prossimità tra due record come la funzione di prossimità tra i corrispondenti attributi dei due record.	
\end{defn}

Consideriamo in prima analisi la misura di prossimità tra due record aventi un solo attributo ed estendiamo successivamente questa analisi a record con più di un attributo.
\begin{figure}[H]
	\centering
	\includegraphics[height=0.3 \linewidth]{clustering/pict/proximity_one.png}
	\caption{Tabella per misurare la prossimità di record con un solo attributo in funzione del tipo di attributo}
\end{figure}

Consideriamo due record riferiti al medesimo attributo nominale qualitativo, tutto ciò che possiamo dire è se i due record hanno  lo stesso valore o meno.
Per quanto riguarda gli attributi binari la dissimilarità esclude la similarità con valori 0 e 1. 

Se ho attributi categoriciordinali assegno un valore intero agli stessi in base alla scala utilizzata e calcolo la dissimilarità come rapporto tra la differenza dei due record e la scala totale di valori di utilizzo. Naturalmente va notato che sto utilizzando una scala linerare, questa è ovviamente un'assunzione molto forte che però bisogna tenere in conto in quanto qualsiasi scala di valori scelta è fatta basandosi su assunzioni.

Come notiamo dalla tabella è molto più facile definire la prossimità tra due attributi numerici, essa è infatti definita come \textit{la differenza in modulo tra i due valori.}
\subsubsection{Misure della distanza}

Quando ho degli attributi numerici posso definire altre misure di distanza nel seguente modo:
\begin{figure}[H]
	\centering
	\includegraphics[height=0.3 \linewidth]{clustering/pict/distanze_minkowski.png}
	\caption{Misure di distanza a partire dalla distanza di Minkowski}
\end{figure}
Ricordiamo che le proprietà che una funzione deve soddisfare per essere definita distanza sono le seguenti:
La misura deve essere
\begin{itemize}
	\item Non negatività: $d(x,y) \geq 0 \quad \forall x,y \quad d(x,y) = 0 \quad iif \quad x = y$
	\item Simmetria $d(x,y) = d(y,x) \quad \forall x,y$
	\item Disuguaglianza triangolare $d(x,z) \leq d(x,y) + d(y,z) \quad \forall x,y,z$
\end{itemize}

La similarità \textit{non rispetta la disuguaglianza triangolare} ma solitamente verifica le proprietà di simmetria e non negatività.

Forniamo ora qualche esempio di misura di prossimità:
\begin{defn}
	 Si definisce \textbf{Simple matching coefficient} la seguente espressione: \[ SMC(x,y) = \frac{\#maching\_attributes}{\#attributes} =  \frac{f_{11}+ f_{00}}{f_{11}+ f_{00} + f_{01}+ f_{10}} \]
\end{defn}
Questa è una misura che ha senso per attributi \textit{simmetrici binari}, ossia per attributi che possono assumere solo i valori 0 o 1 in circa egual quantità.
 Questa misura è invece scomoda se non possiamo affermare se gli 0 siano veramente degli 0, per il valore 1 invece è chiaro. In questo caso si utilizza un'altra misura che è derivata da questa misura, ossia si usa l'assunzione per cui l'osservazione di un evento (identificata con 1), abbia peso maggiore della non osservazione di un evento (rappresentata con 0). Si definisce quindi il \textbf{coefficiente di Jaccard}

\begin{defn}
	Si definisce \textbf{Jaccard Coefficient} la seguente espressione:    \[ J(x,y) = \frac{\#maching\_attributes}{\#attributes\_except00} = \frac{f_{11}}{f_{11}+ f_{01}+ f_{10}}\]
\end{defn}


Questa è ovviamente una misura distorta rispetto agli 1 (infatti è un tipo di misura che va bene con attributi \textit{asimmetrici}) che mi permette però di focalizzare la mia attenzione sulla presenza di questi ultimi. 
Definiamo quindi un nuovo indice:

\begin{defn}
	Si definisce \textbf{Extended Jaccard Coefficient} la seguente espressione: \[ EJ(x,y) = \frac{x \cdot y}{||x||^2 + ||y||^2 - x \cdot y}\]
\end{defn}

 Questa misura è distorta per trattare dati sparsi, quindi tanti elementi in cui ho 0 e solo poche diverse da 0, questo si usa ad esempio nell'analisi del linguaggio naturale. Penso ad esempio ai tweet, una parola che c'è in una frase ha una rilevanza maggiore rispetto ad una parola non presente nella stessa frase.

\subsubsection{Altre misure di prossimità}
Esplicitiamo  ulteriori misure di prossimità.
\begin{defn}
	Si definisce \textbf{cosine similarity.} la seguente espressione:   \[ cos(x,y) = \frac{x \cdot y }{||x|| \cdot ||y||  } \]
\end{defn}

 Viene usata quando tutti gli attributi sono di natura numerica, e si ignorano i match di natura 00. Il vantaggio di questa misura rispetto a quella di Jaccard è che in grado di trattare anche attributi non binari. E' molto utile quindi per comparare record sparsi ed è molto usata in \textit{Information Retrieval} dove i documenti (rappresentati come conteggi di vettori) devono essere comparati.


\begin{defn}
	Si definisce \textbf{Correlazione} la seguente espressione:    \[ corr(x,y) = \frac{cov(x,y)}{std(x)std(y)}\]
\end{defn}
Questa è la stessa correlazione di Pearson ma non legata alle variabili aleatorie.

Elenchiamo ora diverse problematiche legate alle misure di prossimità:
\begin{itemize}
	\item Come si trattano attibuti su scale di ampiezza diverse e/o correlati?
	
	Per risolvere il primo problema faccio la seguente cosa: normalizzo i valori, se ciò non venisse fatto le distanze Euclidee tra i due valori risulterebbero totalmente distorte a favore del valore maggiore. 
	
	Quando gli attributi sono fortemente correlati invece il trucco sta nel fatto che la misura di similarità è molto simile al grado di correlazione tra questi attributi, in tal caso utilizziamo la distanza di Mahalnobis:
	
	\[Mahal(x,y) = (x- y)\Sigma^{-1}(x- y)^{T}\]
	
	Ovviamente questa è una distanza che va usata solo se tutti gli attributi sono numerici.
	\item Come si calcola la prossimità tra record composti da attributi di tipo diverso?
	
	 Per risolvere questo problema mi occupo di valutare tutte le misure di prossimità enunciate in precedenza stando coerenti col tipo di attributo trattato.  
	Dopo averlo fatto uso una varibile indicatrice $\delta_{k}$ per ogni attributo k come segue:
	
	%\delta_{k} = 
	%\begin{cases}
	%0  se il k-esimo attributo è asimmetrico ed entrambi i valori hanno \\  valore 0,  o almeno uno dei record presenta un missing value
	%\\
	%1  altrimenti 
	%\end{cases}

	Una volta definite queste allora la similarità si calcola come:
	\[similarity(x,y) = \frac{\sum_{k=1}^{n}\delta_{k}s_{k}(x,y)}{\sum_{k=1}^{n}\delta_{k}}\]
	\item Come si tratta la prossimità quando gli attributi hanno diversa rilevanza, ossia quando gli attributi contribuiscono secondo pesi diversi all'analisi? 
	
	Per risolvere quest'ultimo problema procedo esattamente nel modo precedente assegnando però dei pesi, le formule risolutive diventano quindi:
	\[similarity(x,y) = \frac{\sum_{k=1}^{n}w_{k}\delta_{k}s_{k}(x,y)}{\sum_{k=1}^{n}\delta_{k}}\]
\end{itemize}

Come è facilmente intuibile risulta molto complesso sostenere tutta questa specificità. Per farlo cerchiamo di ricondurci alla seguente scelta:
\begin{itemize}
	\item Dati densi e continui: distanze metriche  come quella euclidea sono buone rappresentazioni.
	\item Dati sparsi, binari asimmetrici: misure della distanza che ignorano i match 00 come cosine, Jaccard e Extended Jaccard.
\end{itemize}

\subsection{Clustering Algorithms}
Passiamo ora a parlare degli algoritmi di Clustering veri e propri:
\subsubsection{Prototype Based}

L'approccio ai  \textit{Prototype-Based Clustering} si basa sull'assunzione che ogni cluster possa essere ben rappresentato da un unico punto chiamato \textbf{prototipo.} Ogni oggetto è quindi collocato nel cluster del prototipo a cui è più vicino.

\begin{figure}[H]
	\centering
	\includegraphics[height=0.4 \linewidth]{clustering/pict/prototype_cluster.png}
	\caption{Esempio di prototype-based clustering}
\end{figure}
Esistono diversi tipi di algoritmi basati sul prototipo a seconda delle seguenti caratteristiche:

\begin{itemize}
	\item Ogni oggetto deve appartenere ad un singolo cluster.
	\item Ogni record è nella condizione di appartenere a più di un cluster contemporaneamente.
	\item Il concetto di cluster è modellizzato con una distribuzione di tipo probabilistico.
	\item I cluster sono costretti ad avere relazioni fissate.
\end{itemize}

\paragraph{K-medie}

Partiamo ora della descrizione di uno dei primi algoritmi basati sul prototipo. In questo caso il prototipo prende il nome di \textit{centroide,} questo valore solitamente è identificato dal vettore media dei valori degli attributi delle osservazioni di quel determinato cluster. Non siamo vincolati a ragionare in due dimensioni, il cluster generalmente è applicato ad oggetti in uno spazio continuo n-dimensionale.

Forniamo ora una descrizione schematica dell'algoritmo:
\begin{itemize}
	\item Selezione K oggetti come centrodi
	\item \textbf{Repeat} (ciclo)
	\begin{itemize}
		\item Da K cluster, assegniamo ogni record al centroide più vicino.
		\item Ricalcola il centroide di ogni cluster.
		\item \textbf{Until} i centroidi non cambiano.
	\end{itemize}
	
\end{itemize}

Come notiamo va esplicitato il numero di cluster prima dell'esecuzione dell'algoritmo,  valori di partenza differenti possono fornire risultati molto diversi. La scelta del numero di cluster è dunque un ambito da tenere fortemente in considerazione.

L'algoritmo delle K-medie non è vincolato ad utilizzare la distanza Euclidea ma possiamo utilizzare le diverse misure di prossimità utilizzate in precedenza. In particolare:

\begin{itemize}
	\item \textbf{Manhattan:} in questo caso utilizziamo le mediane come centroidi, l'obiettivo è quindi \textit{minimizzare la somma delle $L_{i}$ distanze dei record rispetto al centroide del cluster a cui appartiene.}
	\item \textbf{Squared Euclidea:} in questo caso utilizziamo le medie come centroidi, l'obiettivo è quindi \textit{minimizzare la somma delle $L_{i}$ distanze dei record rispetto al centroide del cluster a cui appartiene.}
	\item \textbf{Cosine:} in questo caso utilizziamo le medie come centroidi, l'obiettivo è quindi \textit{massimizzare la somma delle cosine similarity dei record rispetto al centroide del cluster a cui appartiene.}
\end{itemize}


Elenchiamo tutta una serie di problematiche relative all'utilizzo dell'algoritmo delle K-medie come algoritmo di clustering.
\begin{itemize}
	\item\textit{Scelta dei centroidi iniziali:} è una fase fondamentale e influenza in modo pesante le performance dell'algoritmo in generale. Se optiamo infatti per una scelta random dei centroidi iniziali possiamo avere cluster molto diversi. Vi sono diverse alternative per ovviare a questo problema quali il clustering gerarchico.
	\item \textit{Complessità spaziale e temporale:} questi sono due punti a favore del K-medie. In particolare occupa pochissimo spazio in quanto vengono salvate solo le posizioni dei centroidi. Inoltre si tratta di un algoritmo abbastanza rapido in quanto è lineare rispetto al numero di istanze considerate. 
	\item \textit{Cluster vuoti:} bisogna tenere in considerazione che può capitare che. un cluster sia vuoto per scelta sbagliata di centroidi iniziale (magari randomica).
	\item \textit{Presenza di outlier:} Questi elementi creano grossi problemi nel calcolo della media delle osservazioni di un cluster. Risulta efficiente però nella ricerca di outlier in quanto verranno identificati come cluster di singleton. Risulta spesso utile rimuoverli.
\end{itemize}

Elenchiamo ora una serie di limiti riferiti alla ricerca dei cluster con l'algoritmo delle k-medie:
\begin{itemize}
	\item Diffiicoltà  a ricercare cluster di non forma sferica. 
\begin{figure}[H]
	\begin{minipage}[b]{0.47\textwidth}
		\centering
		\includegraphics[width=\textwidth]{clustering/pict/non_sferica_1.png}
		\caption{Cluster non sferico.}
	\end{minipage}
	\hfill
	\begin{minipage}[b]{0.47\textwidth}
		\centering
		\includegraphics[width=\textwidth]{clustering/pict/non_sferica_2.png}
		\caption{Cluster sferico.}
	\end{minipage}
\end{figure}
	
	\item Difficoltà a trovare cluster con dimensioni diverse, questo problema nasce dal fatto che la distanza è fissata.
	\item difficolt\`a a indentificare cluster di diversa densit\`a (vicinanza dei punti)
\end{itemize}


\begin{comment}

Per ovviare ad alcuni di questi problemi vi sono tecniche per clusterizzare in cluster pi\`u numerosi ed alla fine unire secondo degli algoritmi specifici nel numero di cluster richiesto, es minimizzazione delle distanze dei centroidi. 

Quindi K-medie \`e:
\begin{itemize}
	\item semplice e si applica a tanti tipi di dati
	\item \`e abbastanza efficiente anche se vengono effettuate pi\`u applicazioni successive dello stesso
	\item vi sono varianti pi\`u efficienti e meno problmatiche (es. bisecting k-means)
	\item non adatto a tutti i tipi di dat e non pu\`o gstire cluster non sferici, con size e densit\`a diverse
	\item problemi con dati outliers
	\item \`e strettamente legato al concetto di centroide. Vi sono delle varianti che utilizzano il \underline{medoide} ed \`e pi\`u efficiente
	
	
\end{itemize}
NB: pi\`u cluster vengono utilizzati pi\`u aumenta la complessit\`a computazionale.


\paragraph{Fuzzy C-means}
Se gli elementi sono distribuiti in gruppi ben separati il clustering \`e semplice, i problemi nascono dove le istanze sono molto vicine. Inoltre consideriamo il fatto di osservazione sulla frontiera delle distanze dei cluster. 

Per ovviare al problema ogni osservazione viene considerata come appartenente ad ogni cluster ma con una misura diversa per ognuno! (es. utilizzando la misura della media)

$w_{ij}$ \`e il \textbf{peso} con cui l'$i$-esimo oggetto appartiee al $j$-esimo cluster.

La \textbf{pseudo-partizione Fuzzy} \`e definita assegnando prima l'isieme di tutti i pesi $w_{ij}$ con $w_{ij} \geq 0$ ed essi devono essere verificati dai seguenti vincoli:

\[\sum_{j=1}^{K} w_{ij} = 1\] con $i = 1, ..., m$

\[ 0 < \sum_{i=1}^{m}w_{ij} < m\] con $j = 1, ..., K$

Ogni oggetto deve essere assegnato ad un certo grado a tutti i cluseter. Non sono permessi cluster vuoti e ogni oggetto pu\`o essere assegnato esclusivamente ad un singolo cluster.

Il Fuzzy C-means produce cluseter che danno un'indicazione del grado in cui ogni oggetto appartiene ad ogni cluster, ha gli stessi punti di forza e debolezza del K-means ma pi\`u pesante computazionalmente.

\paragraph{Modelli a mistura}
In questi modelli vediamo le osservazioni da una mistura di diverse distribuzioni di problabilit\`a. In sostanza possiamo vedere le misture come distribuite secondo delle normali con uguale varianza ma medie diverse (curve di livello) [propriet\`a rilassabile]. 

Supponiamo spazio di 3 componenti, processo di generazione:
\begin{enumerate}
	\item Seleziono uno delle 3 componenti
	\item Estraggo un campione dalla componente selezionata
	\item ripeto 1 e 2 m volte per ottenere il dataset
\end{enumerate} 

\[ p(\bar{x}|\Theta) = \sum_{j=1}^{K} w_j p(\bar{x}|\theta_j)\] probabilit\`a dell'oggetto x dove $w_j$ probabilit\`a della j-esima componente, $p(\bar{x}|\theta_j)$ probabilit\`a di estrarre x dalla j-esima component, $\theta_j$ parametri associati a j.

Il processo sostanzialmente \`e quello di partire dai dati e ridurrli nelle misture significative. (simile a come avviene la nostra capacit\`a di isolare il rumore uditivo). Vi \`e un algoritmo utilizzato per la risoluzione che \`e l'\textbf{Expectation Maximization}. Si tratta di una classe di algoritmi che sono diversi in base alla loro applicazione. 

Svantaggi:
\begin{itemize}
	\item Apprendimento \`e lento
	\item non \`e pratico per i modelli con un gran numero di componenti
	\item non funziona bene quanto i cluster conengono  poche osservazioni
	\item non funziona bene quando gli oggetti sono co-lineari
\end{itemize}

\begin{itemize}
	\item pi\`u generico di k-means e fuzzy c-means perch\`e usa distribuzioni di vario tipo
	\item disciplina l'approccio di eliminazione di complessit\`a
\end{itemize}


\paragraph{Mappe di Kohonen o SOMs}
Una struttura feedforward per algoritmi di clustering. Detta anche mappa atorganizzante. Vi \`e una mappatura dei neuroni che mappano in uno spazio bidimensionale. Non hanno una sola direzione ma sono bi-direzionali, possono spostare i pesi sia da input ad output che viceversa.

Ho uno spazio di input continuo e voglio mapparlo in uno spazio discreto di output. Vorrei che mappando un'osservazione in un neurone posso poi tornare indietro tramite un altro collegamento.

Vi \`e un'informazione topografica in tutto ci\`o. Ed \`e una cosa unica rispetto agli altri algoritmi di clustering considerati. Durante la fase di apprendimento sfruttiamo la topologia considerando ogni oggetto aggiornato al centroide pi\`u vicino secondo essa. Considerando un neurone assegnato ad una osservazione, allora anche i neuroni ad esso adiacenti (vicinato - definito in vari modi) dovrebbero risentirne di questo assegnamento, in questo modo vengono trainati i neuroni e i vicini degli stessi. 

Riassumendo:
\begin{enumerate}
	\item inizializzazione, seleziono i centroidi assocaiti a ciascun neurone
	\item competizione tra diversi neuroni per ottenere l'istanza. Ogni neurona fa un'offerta diversa per vincere quell'oggetto e la proposta \`e proporzionata in base a quanto un neurone (in base al suo centroide) si considera "vicino" all'istanza considerata, viene poi proclamato il vincitore e gli viene assegnata l'istanza (in base a una misura di performance)
	\item collaborazione una volta che un neurone vince distribuisce il suo vantaggi ai vicini (processo di premi-punizioni) aggiustando i centroidi
	\item aggiornamento del centroide del vincitore e dei vicini calcolandol attraverso l'istanza appena considerata
\end{enumerate} 

Punti di forza Interpretabilit\`a dei cluster 

Limiti:
\begin{itemize}
	\item bisogna scegliere un gran numero di parametri, funzione di vicinato ecc e varia molto nelle performance
	\item non sempre un raggruppamento identifica un singolo cluster naturale (solitamente dopo viene applcato un k-medie sui centroidi trovati)
	\item difficile paragonare i risultati
	\item la convergenza non \`e garantita, \`e un'euristica
\end{itemize}

\paragraph{Clustering Gerarchico}
\begin{itemize}
	\item \textbf{agglomerativo}: si parte da tutti gli oggetti come cluster individuali e ad ogni step unisce le coppie di cluster pi\`u vicine.
	\item \textbf{divisivo}: parte da un cluster unico e passo passo divide il cluster in soli cluster singleton di oggetti individuali rimanenti. Abbiamo bisogno di decidere quale cluster splittare ad ogni passo e come splittare
\end{itemize}

\textbf{Clustering gerarchico agglomerativo} sono i pi\`u comuni. La soluzione che si ottiene \`e il \textbf{dendrogramma} del clustering effettuato. \`E computazionalmente molto pesante.

\begin{figure}[h!]
	\centering
	\includegraphics[height=0.45 \linewidth]{clustering/pict/cluster_aggl.png}
	\caption{da clustering gerarchico a dendrogramma}
\end{figure}

Si parte dai singoli elementi e pian piano vengono generati cluster sempre pi\`u inclusivi fino ad ottenere un unico cluster. \\

\noindent
\textbf{Passaggi}:
\begin{enumerate}
	\item Calcolare la matrice di prossimit\`a
	\item \textbf{Ripeti}
	\item Unione di due cluster vicini
	\item Aggiorna la matrice di prossimit\`a (o delle distanze) che riflette la prossimit\`a tra il nuovo cluster e il cluster originale
	\item \textbf{Finch\`e} rimane un solo cluster
\end{enumerate}
\noindent
La prossimit\`a tra cluster viene calcolata con:
\begin{itemize}
	\item  \textit{Min or Single Linkage}: minore distanza tra tutte le possibili coppie di elementi presenti nei due cluster (soluzione pi\`u ottimista)
	\item \textit{Max or Complete Linkage}: maggiore distanza tra tutte le possibili coppie di elementi presenti nei due cluster (soluzione pi\`u robusta o pessimista)
	\item \textit{Group Avarage or Avarage Linkage}: media distanza tra tutte le possibili coppie di elementi presenti nei due cluster (soluzione media - politically correct cit)
	\item \textit{metodo di Ward}: assume i cluster come rappresentati dai centroidi, misura la prossimit\`a tra due cluster come somme di scarti quadratici che risulta dalla fusione di due cluster
\end{itemize}
\noindent
\textit{Caratteristiche}:
\begin{itemize}
	\item Non risolve problemi di ottimizzazione globale
	\item pu\`o gestire cluster con dimensioni differenti, gli elementi possono anche essere pesati o meno
	\item le decisioni di unione sono definitive: non si pu\`o tornare indietro rispetto alla decisione (approccio greedy)
\end{itemize}
\noindent
\textit{Difetti}:
\begin{itemize}
	\item computazionalmente molto pesante e richiede molto spazio
	\item decisioni di unione di cluster sono subottimali quindi creano rumore
\end{itemize}

\paragraph{Density-Based Clustering}
Densit\`a di regioni di oggetti che sono circondati da regioni con bassa densit\`a di oggetti.

\textbf{DBSCAN} \`e un semplice ma efficace metodo density-based. 

Vi sono diversi metodi per definire la \textit{densit\`a}, descriviamo il center-based approach: la densit\`a \`e il numero di oggetti all'interno di uno specifico raggio (\textbf{Eps}) di quell'oggetto. 

\noindent
Ci sono 3 tipi di punti:
\begin{itemize}
	\item \textbf{core point}: stanno all'interno della regione ad elevata densit\`a. \`E core pont se il numero di punti nel vicinato attorno ad esso eccede un certo \textit{threshold} MinPts, impostato dall'utente
	\item \textbf{border point} (wanna be core point): un punto che include nel suo vicinato un core point, quindi lui si considera un vicino di core point
	\item \textbf{noise point} (osservazione residuale no core o border): non \`e ne un core point ne un border point
\end{itemize}

\noindent
\textbf{Passaggi}
\begin{enumerate}
	\item Etichettare tutti i core, border, noise point
	\item Si eliminano i noise point
	\item collegano i core point che sono all'interno dei rispettivi Eps
	\item si fa ogni gruppo di core point connessi in cluster separati
	\item si assegna ciascun border point in un cluster associato ai core point pi\`u prossimo
\end{enumerate}

La scelta dei parametri \textit{MinPts} e \textit{Eps} \`e fondamentale.
\\
\\
\underline{DBSCAN vs K-medie}:
\begin{itemize}
	\item DBSCAN and K-Means assign objects to a single cluster, but K-Means assigns all
	objects while DBSCAN can discard noise objects
	\item DBSCAN can handle clusters of different sizes and shapes and it is not strongly
	affected by noise or outliers. K-means has difficulties with non-globular clusters and
	clusters of different sizes. Both algorithms perform poorly when clusters have widely
	differing densities
	\item K-Means can only be used for data that has a well defined centroid, such as mean or
	median. DBSCAN requires that its definition of density, which is based on the
	traditional Euclidean notion of density be meaningful for the data
	\item K-Means can be applied to sparse, high-dimensional data, such as document data.
	DBSCAN typically performs poorly for such data because the traditional Euclidean
	definition of density does not work well for high-dimensional data
	\item DBSCAN makes no assumption about the distribution of the data. The basic K-means
	is equivalent to a statistical clustering approach (Mixture Model) that assumes all
	clusters come from spherical Gaussian distributions with different means but the
	same covariance matrix
	\item DBSCAN and K-Means both look for clusters using all attributes, that is, they do not
	look for clusters that may involve only a sub-set of the attributes
	\item K-Means can find clusters that are not well separated, even if they overlap, but
	DBSCAN merges clusters that overlap
	\item K-Means has complexity O(m) while DBSCAN is O(m2) (except for low dimensional
	Euclidean data.)
	\item DBSCAN produces the same set of clusters from one run to another while K-Means,
	which is typically used with random initialization of centroids, does not
	\item DBSCAN automatically determines the number of clusters, for K-Means the number
	of clusters needs to be specified as a parameter
\end{itemize}
\noindent
Vi sono altri algoritmi di density-based:
\begin{itemize}
	\item Grid-based clustering
	\item subspace clustering
	\item kernel density function
\end{itemize}

\paragraph{Graph-based Clustering Algorithm}
Gi\`a gli algoritmi a rappresentazione gerarchica possono essere visti come graph-based. 

Sparsificare: tagliare, rimuovere certi archi che non superano la verifica di una certa condizione (Es. tutti gli archi che non superano una certa soglia, oppure un concetto di vicinato). 

Definire una misura di similarit\`a tra due oggetti basati sul numero di nearest neighbors. 

Definire oggetti core e costruire cluster attoro ad essi. Necessita di concetti di densit\`a e prossimit\`a.

Utilizza l'informazione nel grafo di prossimit\`a per fornire una maggiore valutazione di specialit\`a in cui due cluster dovrebbero essere uniti. 

Grafo di prossimit\`a stabilisce dei pesi tra i vari nodi che \`e il costo da percorrere tra un nodo ed un altro, \`e rappresentazione della matrice delle distanze. 

I legami di similarit\`a tra nodi vengono suddivisi in \textit{sufficientemente vicini} (superano il threshold) e \textit{non vicini} quindi con vicinanza minore rispetto al threshold. Vengono quindi rimossi gli archi che non superano il threshold, questo significa \textit{sparsificare} la matrice. Viene cos\`i generato il grafo di prossimit\`a sparsificato.

Definiamo quindi il \textbf{K-Nearest Neighbor}\\
se fisso k = 3 scelgo per ogni riga della matrice delle prossimit\`a i primi 3 valori degli archi per come vicinato. 

2 modi di procedere: threshold o k neighbors.

La sparsificazione riduce anche di molto il numero di elementi del dataset. Il clustering lavora molto meglio in quanto si lavora in un concetto di vicinanza. Algoritmi su grafi partizionati possono essere usati. 

\paragraph{Altri algoritmi}
Non sempre purtroppo la sparsificazione del grafo porta ad un grafo sconnesso. L'idea \`e di ciclare la sparsificazione finch\`e non otterr\`o dei grafi partizionati ben definiti. 

\textbf{Minimum spanning tree} o albero di supporto a costo minimo:
\begin{itemize}
	\item non ha cicli
	\item contiene tutti i nodi dell'albero
	\item ha il minimo costo totale dei pesi associati ai suoi alberi di tutti i possibili spanning tree
\end{itemize}
\noindent
\textbf{MST divisive hierarchical clustering Algorithm}
\begin{enumerate}
	\item Calcola il MST per grafo di dissimilarit\`a
	\item \textbf{Ripeti}
	\item Crea un nuovo cluster per rompere il collegamento corrispondente alla pi\`u grande dissimilarit\`a
	\item \textbf{Finch\`e} rimangono solo cluster singoli
\end{enumerate}
\noindent
Algoritmo \textbf{Opossum}:
\begin{enumerate}
	\item Caclola il grafo di similarit\`a sparso
	\item partiziona il grafo in k componenti distinte (cluster) usando METIS
\end{enumerate}
\noindent
Altro algoritmo \textbf{Camaleon}.

Quando si visualizza il dendrogramma, si possono visualizzare le distanze e solitamente ha una forma a "gomito" la soluzione ottimale \`e solitamente quella che precede la salita finale del comito. Un buon dendrogramma ha grandi salti prima di aggregare insieme (salti grandi significa che se raggruppo genero un errore importante quindi \`e positivo per la valutazione del clustering). 

\subsection{Clustering Evaluation}

L'efficienza di un algritmo di clustering viene valutato tramite:
\begin{itemize}
	\item similarit\`a
	\item esclusivit\`a dei clustering o misure fuzzy di sovrapposizione
	\item completo o parziale clustering
\end{itemize}

Innanzitutto scelgo alcuni algoritmi candidati da voler implementare in base alla tipologia di dati che ho. L'utilit\`a di utilizzarne diversi e poterli confrontare. Bisogna impostare un'insieme di esperimenti equi per poterli valutare correttamente (stessi dati, chiare differenze di parametri ecc). 
Il problema \`e che \`e molto complesso valutare i cluster visto che non so esattamente cosa sto cercando. 

I problemi grossi nella valutazione sono:
\begin{itemize}
	\item determinare la tendenza dei cluster (che non siano fatti a caso)
	\item determinare il numero corretto di cluster (qualunque cosa significhi)
\end{itemize} 

abbiamo 3 diversi tipi di indici:
\begin{itemize}
	\item \textbf{Esterni o supervisionati}: misure che estendono i cluster scoperti con alcune strutture esterne (sapendo circa la verit\`a sul problema)
	\item \textbf{Interni o non supervisionati}: misure di similarit\`a di cluster in particolare le strutture in base a informazioni esterne, possono essere:
	\begin{itemize}
		\item Misure di coesione: determina la connessione tra elementi di un cluster
		\item misure di separazione: quanto sono ben distiniti i cluster 
	\end{itemize}
	\item \textbf{Relativi}: compara differenti clustering
\end{itemize}

\subsubsection{Esterni o supervisionati}
$P = \{P_1, ..., P_R\}$ partizione di un dataset di m (oggetti record) in R categorie, possono essere partizionati con algoritmi in k cluster $C= \{C_1, ..., C_K\}$.

Casi:
\begin{enumerate}
	\item (\textbf{a}) x e y appartengono allo \underline{stesso} cluster di C e alla \underline{stessa} categoria di P
	\item (\textbf{b}) x e y appartengono allo \underline{stesso} cluster di C e a \textit{differenti} categorie di P
	\item (\textbf{c}) x e y appartengono a \textit{differenti} cluster di C e alla \underline{stessa} categoria di P
	\item (\textbf{d}) x e y appartengono a \textit{differenti} cluster di C e a \textit{differenti} categorie di P
\end{enumerate}

Numero totale di coppie che si possono formare:
\[ M = \frac{m x (m-1)}{2} = a + b + c + d\]

Rand: \[R = \frac{a + d}{M}\] con $R \in [0,1]$

Jaccard: \[J = \frac{a}{a + b + c}\] con $J \in [0,1]$

Fowlkes and Mallows: \[FM = \sqrt{\frac{a}{a + b} x \frac{a}{a + c}}\]

$\Gamma$ statistics

\subsubsection{Interni o non supervisionati}

Molte misure di validit\`a del partizionamento dei cluster si basano sulla \textbf{coesione} e sulla \textbf{separazione}. Noi di base vogliamo un valore di coesione alto ed un valore di separazione basso.

Ci si basa sul concetto di grafo (graph-based), la coesione di un cluster pu\`o essre definita come a somma dei pesi degli archi inn un grafo di prossimit\`a che connette i punti all'interno del cluster. 

\[ cohesion(C_i) = \sum_{x, u \int C_i} proximity(x,y) = \sum_{x y \in C_i} similarity(x,y)\]

La similarit\`a \`e quindi inversamente proporzionale alla dissimilarit\`a. 
La separazione \`e valutata secondo la somma dell prossimit\`a di nodi che non appartengono allo stesso cluster.

Per i cluster prototype-based la coesione \`e valutata rispetto al centroide che ne rappresenta il prototipo; la separazione \`e misurata sulla prossimit\`a del centroide o prototipo del cluster. Altro metodo per la separazione \`e la valutazione rispetto al centroide generato dai due cluster.

La validit\`a \`e valutata in questo modo: $overall validity = \sum_{i = 1}^{K} w_i \cdot validity(C_i)$, dobbiamo decidere per\`o il peso da applicarvi ($W_i$). 


Importante capire che alle volte separare un cluster potrebbe migliorare le misure di coesione, quando ho valori di coesione bassi.

Se invece ho valori di coesione buoni ma valori di separazione alti allora provando ad unirli si raggiungono cluster pi\`u coesi. 

La valutazione che sta all'interno del cluster contribuisce maggiormente informazione alla coesione e separazione, invece per le osservazioni pi\`u esterne queste valutazioni possono non essere cos\`i utili. Pertanto \`e stato definito l'\textbf{indice di Silhouette}. questo indice combina coesione e separazione per l'i-esimo oggetto e si definisce in questo modo:

\[s_i = \frac{b_i - a_i}{max(a_i,b_i)} \in [-1,+1]\]

dove $a_i$ \`e la media della distanza tra l'oggetto i-esimo e gli altri oggetti nel cluster, $b_i$ la minima distanza media dell'oggetto i-esimo a tutti gli altri oggetti di ogni custer diverso dal cluster di partenza dell'oggetto.

(es. se $a_i = 0$ allora $s_i = 1$ perch\`e si trova nel cluster migliore. se ha invece valore $< 0$ significa che se avessi assegnato ad un altro cluster avrei avuto un indice di silhouette pi\`u alto)

L'avarage Silhouette coefficient di un cluster da un valore di silhouette medio 

Per il clustering gerarchico viene definito il Cophenetic Correlation Coefficient: 
Si considera la matrice di prossimit\`a e la matrice cophenetica.
Cophenac matrix \`e la matrice che tiene conto delle distanze tra coppie di osservazioni raggruppate all'interno di uno stesso cluster. Il valore varia da $[-1,1]$. Pi\`u il valore \`e vicino a 1 meglio l'algoritmo gerarchico fitta.

\subsubsection{Paradigma di validit\`a}
Rispondere alla domanda se il nostro dataset tende ad essere clusterizzabile.

Vengono considerati criteri interni ed esterni legati a metodi statistici e test di ipotesi. 

Se non ci fosse nessuna struttura dei dati (ipotesi NULLA). 

\begin{itemize}
	\item Random Position Hypothesis in tutte le location di m osservazioni in specifiche regioni n-dimensionali sono equamente buone $\rightarrow$ per ratio data
	\item Random Graph Hyothesis la matrice di prossimit\`a ha rank equamente buoni $\rightarrow$ per prossimit\`a ordinali tra coppie di osservazioni
	\item Random Level Hypothesis tutte le permutazioni delle label di m osservazioni sono equamente buone $\rightarrow$ per tutti i tipi di dati
\end{itemize}

A questo punto valuto in base alla distribuzione sotto l'ipotesi nulla e ci calcolo l'indice. Se il valore dell'indice \`e all'interno del quantile fissato ($\alpha$) posso affermare che ci sia struttura nel dataset, se invece l'ipotesi \`e rigettata allora si sa che il clustering sviluppato non ha struttura pertanto pu\`o essere mostrata al proprio interlocutore l'algoritmo sviluppato. 

\subsubsection{Selezione del numero di cluster}
A senso porsi questa domanda se si \`e prima rilevato che vi \`e struttura di clustering. 

I criteri \textbf{relativi} \`e un uso particolare delle misure di clsustering per comprendere il numero ottimale di cluster. Si proietta lo spazio delle osservazioni in uno spazio di dimensione minore e da questo si utilizzano delle tecniche di valutazione.

Gli indici principali sono:
\begin{itemize}
	\item Calinski and Harabasz: seleziono il numero di k che ottimizza il numero di cluster
	\item Dunn: simile a quello sopra
	\item Devies-Bouldin: minimizzo K per l'ottimo numero di cluster
\end{itemize}

Misure miste di probabilit\`a model-based clustering sono:
\begin{itemize}
	\item Akaike Information Criterion (AIC)
	\item Minimum Description Length (MDL)
	\item Bayesian Information Criterion (BIC)
\end{itemize}

Per l'algoritmo di clustering richiede un input di numero di clsuter K dagli utenti una sequenza di strutture di clustering ottenute eseguendo l'algoritmo motlre volte dove K spazia da $K_{min}$ a $K_{max}$.

\end{comment}
\clearpage


\section{Association Analysis}
\subsection{Introduction (*)}

L'analisi di associazione riporta al concetto di causalità di variabili: dati certi valori di attributi cosa posso dire del valore di un altro attributo?

Le regole associative permettono di prendere delle scelte molto operative in diversi ambiti, in particolare in quello che viene chiamato \textit{market busket analysis}. 
Il problema del carrello tratta il posizionamento di prodotti in un negozio. Si sa che all'acquisto di un certo prodotto si tende a acquistare altri prodotti connessi, quindi si cercano queste associazioni basandosi sul carrello della spesa dei clienti (appunto market basket) per poi capire come impostare la disposizione sugli scaffali.

\textbf{Obiettivo}: identificare quali siano gli \textbf{item associati} per poter prendere delle decisioni. In sostanza si generano \textbf{Regole associative} formate da coppie di insiemi \textit{antecedente} e \textit{conseguente}.

es. \{Beer\} $\rightarrow$ \{swiss cheese\} 

\quad \{antecedente\} $\rightarrow$ \{conseguente\}\\
\textbf{NB}: non \`e una causalit\`a ma un'associazione

L'analisi si basa sullo studio di due diversi dataset:
\begin{itemize}
	\item Product set: contiene informazioni legate ai prodotti come il nome e il prezzo
	\item Transaction set: contiene informazioni legate agli acquisti dei clienti, ogni record corrisponde ai prodotti presenti nel carrello del cliente
\end{itemize}

Si organizza il dataset delle transazioni in formato binario, ovvero ogni colonna indica se in una data transazione un certo prodotto sia o meno presente.

\begin{figure}[H]
	\centering
	\includegraphics[height=0.25 \linewidth]{association/pict/transaction_set_bin.png}
	\caption{esempio: transaction set binarizzato}
\end{figure}
\clearpage
\noindent
Consideriamo:

$I = \{i_1, i_2,...,i_d\}$ il set di tutti gli item nel market basket

$T = \{t_1,t_2,...,t_N\}$ l'insieme di tutte le transazioni

\begin{defn}
	Una collezione di zero o più item è chiamata \textbf{Itemset}.
\end{defn} 
\begin{defn}
	Se un itemset contiene k item \`e detto \textbf{K-Itemset}.
\end{defn}
\begin{defn}
	L'itemset che non contiene alcun elemento è detto \textbf{empty set}.
\end{defn}
\begin{defn}
	La \textbf{transaction width} è il numero di item presenti in una transazione
\end{defn}
	 
Una transazione $t_j$ contiene l'itemset $X$, se $X$ è un sottoinsieme di $t_j$.

\begin{defn}
	Il \textbf{Support count} è il numero di transazioni che contentono uno specifico itemset.

\[ \sigma(X) = |\{t_i | X \subseteq t_i, t_i \in T\} \]
\end{defn}

\begin{defn}
Una \textbf{regola di associazione} viene rappresentata come:

$X \rightarrow Y$

dove $X$ e $Y$ sono itemset disgiunti ($X \cap Y = \emptyset$).
\end{defn}
Una regola viene valutata in termini di \textit{supporto} e di \textit{confidenza}.

\begin{defn}
	\textbf{Support} determina quanto spesso una regola sia applicabile dato un data set:
	
	\[s\{x \rightarrow Y\} = \frac{\sigma(X \cup Y)}{N}\]
\end{defn}
\begin{defn}
	\textbf{Confidence} determina quanto frequente Y \`e presente in una transazione che contiene X (si assume che l'universo sia rappresentato partendo da X):
	
	\[c\{x \rightarrow Y\} = \frac{\sigma(X \cup Y)}{\sigma(X)}\]
\end{defn}

\textbf{Es}. \\
$X = \{swiss cheese, cheddar\}$, $Y = \{diet coke\}$\\
Assumiamo che:
\begin{itemize}
	\item $\sigma(x) = 8$ (support count)
	\item $N = 20$ (numero di transazioni)
	\item $\sigma(x,y) = 6$ 
\end{itemize}
Allora il support e la confidence della regola $X \rightarrow Y$ sono:

\[s\{X \rightarrow Y\} = \frac{\sigma(X \cup Y)}{N} = \frac{6}{20} = 0.3\]

\[c\{x \rightarrow Y\} = \frac{\sigma(X \cup Y)}{\sigma(X)} = \frac{6}{8} = 0.75\]

Perché utiliziamo il supporto:
\begin{itemize}
	\item se troppo basso potrebbe esserci un'associazione casuale
	\item potrebbe non valere la pena seguire associazioni che si applicano in modo poco significativo dal punto di vista dei profitti 
\end{itemize}
Il concetto di supporto è utilizzato per eliminare regole non desiderate e condivide interessanti proprietà che possono essere sfruttate per la ricerca di regole associative efficaci.

La confidenza \`e molto importante perché misura l'affidabilità dell'inferenza e:
\begin{itemize}
	\item un' alta confidenza significa che Y sarà molto presente in transazioni con X 
	\item si stima la probabilit\`a condizionata di Y dato X
\end{itemize}
\begin{defn}
\textbf{Association Rule Mining Problem} può essere formalmente definito come: dato un set di transazioni T, trovare tutte le regole con $support \ge minsup$ e $confidence \ge minconf$, dove $minsup$ e $minconf$ sono i threshold corrispondenti alle due misure.
\end{defn}

\textit{Approccio a forza bruta} di association rule non è molto praticabile in quanto i tempi di computazione aumentano in modo esponenziale: 

$R = 3^d - 2^{d+1} + 1$.

\begin{figure}[H]
	\centering
	\includegraphics[height=0.4 \linewidth]{association/pict/brute_force.png}
	\caption{numero di regole calcolate in base al numero di itemset}
\end{figure}

Una strategia comunemente adottata in molti algoritmi è quella di decomporre il problema in 2 grandi supertask:
\begin{itemize}
	\item\textit{Generazione dei frequenti itemset}: specifichiamo tutte e sole quelle regole per cui il supporto è maggiore del $minsup$, gli itemset generati sono chiamati \textbf{Frequent Itemset}
	\item \textit{Generazione delle regole}: estriamo tutte le regole con alta confidenza (maggiore di $minconf$) dai Frequent Itemset trovati precedentemente, queste regole vengono chiamate \textbf{Strong Rules}
\end{itemize}
\noindent
La complessità maggiore è richiesta dalla generazione dei Frequent Itemset.


\subsection{Rule Extraction}

Per comprendere l'inefficienza della generazione con l'approccio forza bruta pensiamo a questo esempio:
\begin{figure}[H]
	\hspace{-0.7 cm}
	\includegraphics[height=0.5 \linewidth]{association/pict/k-itemset.png}
	\caption{k-itemset brute-force}
\end{figure}
\begin{defn}
	\textbf{Candidate Itemset}: l'insieme di tutti gli itemset che possiamo formare. Avranno un numero di item diverso. 
\end{defn}
Nel nostro caso il numero di itemset candidato è: $M = 2^{d} - 1 = 2^5 -1 = 31$

Come si può notare abbiamo un sistema a doppio cono che è tipica delle distribuzioni binomiali. Una volta che gli abbiamo considerati tutti ci interessano solo i più frequenti. \\
Se usassimo la forza bruta dovremmo calcolare per ogni itemset candidato il suo support count, e vedere se il suo supporto lo configura come un itemset frequente (molto dispendioso). 

I confronti da effettuare sono nell'ordine di $O(NMw)$, dove:
\begin{itemize}
	\item $N$ = numero di transazioni
	\item $M$ = numero di itemset candidato
	\item $w$ = massima lunghezza delle transazioni
\end{itemize} 

È decisamente troppo come numeri confronti contando che molti dei quali sono inutili o poco significativi.

Vi sono due approcci per ridurre il costo computazionale della generazione di itemset frequenti:
\begin{itemize}
	\item ridurre il numero di candidati itemset ($M$). Il principio Apriori \`e un metodo per eliminare alcuni candidati itemset senza contare il support count. 
	\item riduce il numero di confronti anzich\`e controllare tutte le possibili combinazioni, lo si fa con strutture dati avanzate
\end{itemize}

\paragraph{Principio Apriori} se un itemset \`e frequente, allora tutti i suoi sottoinsiemi sono frequenti.\\
Quindi se una regola ha una frequenza bassa allora tutte le regole che prevedono come sottoinsieme la stessa non supereranno quella frequenza pertanto \`e inutile considerarle. Si procede attraverso il \textbf{pruning} dell'albero delle sequenze per queste soluzioni, viene chiamato \textbf{support-based pruning} (vedi immagine).

\begin{figure}[H]
	\centering
	\includegraphics[height=0.5 \linewidth]{association/pict/pruning.png}
	\caption{esempio di support-based pruning}
\end{figure}

\subsubsection{Algoritmo apriori}
Genera due operazioni:
\begin{itemize}
	\item \textbf{Candidate Generation}: genera nuovi candidati k-itemset basati su (k-1)-itemset frequenti calcolati nella precedente iterazione
	\item \textbf{Candidate Pruning}: questa operazione elimina alcuni candidati k-itemset usando la strategia del support-based pruning
\end{itemize}

La complessit\`a computazionale soffre di 4 limiti:
\begin{itemize}
	\item \textit{Support threshold}: la soglia di supporto se troppo bassa non taglio molto l'albero, però non deve essere neanche troppo elevata altrimenti non considero associazioni rilevanti. 
	\item \textit{Numero di item (dimensionalit\`a)}: se il numero di item cresce, ci sar\`a bisogno di più spazio in memoria per registrare il support count degli item, inoltre bisogna considerare anche il costo dell'I/O per passare i dati.
	\item \textit{Numero di transazioni}: l'algoritmo scorre più volte tutta la lista di transazioni, pertanto un numero alto di transazioni inficia sui tempi.
	\item \textit{Avarage transaction width}: per dataset densi la lunghezza media delle transazioni tende ad essere grande. La massima lunghezza degli itemset frequenti tende ad aumentare quindi più sequenze candidato devono essere esaminate durante la generazione e support counting. In aggiunta aumenta il numero di archi traversi nell'albero durante il support counting.
\end{itemize}

\subsection{Maximal/Closed Frequent Itemsets}
\subsubsection{Rule Generation}
Ogni k-itemset frequente, Y pu\`o generare al limite $2^k-2$ regole di associazione. 

Una regola di associazione pu\`o essere estratta partizionando l'itemset Y in 2 sottoinsiemi non vuoti $\{X\}$ e $\{Y-X\}$, tale che $X \rightarrow Y -X$ soddisfa il threshold di confidenza.

\textbf{NB}: Tutte le regole generate da itemset frequenti sono esse stesse frequenti.

In pratica, il numero di itemset frequenti prodotti da transazioni possono essere molto grandi. È utile identificare itemset rappresentativi e piccoli con i quali derivare gli itemset grandi. Per questo si ragiona in due rappresentazioni:
\begin{enumerate}
	\item Maximal Frequent Itemset
	\item Closed Frequent Itemset
\end{enumerate}

\begin{defn}
	\textbf{Maximal Frequent Itemset} è definito come un Itemset Frequente per il quale nessuno dei suoi soprainsiemi immediati sono frequenti.
\end{defn}
\textbf{Maximal Frequent Itemset}: per ogni nodo si verifica il vincolo della frequenza e si definisce la frontiera dove un nodo non gode pi\`u di questa propriet\`a, corrisponde alla massima frontiera in cui mi posso spingere.

\begin{figure}[H]
	\centering
	\includegraphics[height=0.7 \linewidth]{association/pict/max_freq_itemset.png}
	\caption{esempio di frontiera di maximal frequent itemset}
\end{figure}

\textbf{Nella pratica}: Un nodo \`e un Maximal Frequent Itemset se \`e frequente e se tutte le sue estensioni non sono frequenti.\\
\begin{itemize}
	\item Fornisce una rappresentazione compatta dell'insieme di itemset che cerchiamo, la pi\`u piccola espressione in cui gli itemset sono derivabili
	\item Calcolato dal più piccolo insieme di itemset dal quale tutti gli itemset frequenti possono essere derivati
	\item È praticabile solo l'algoritmo efficiente usato esplicita la ricerca dei maximal frequent itemset senza numerare tutti i suoi sottoinsiemi
\end{itemize}
Per costruzione \textit{per\`o} non ci dice quanto \`e il supporto rispetto ai suoi sottoinsiemi. In alcuni casi potrebbe servire avere una minima rappresentazione degli itemset frequenti che preservano l'informazione sul supporto.

\begin{defn}
	Un itemset X è \textbf{Closed Frequent Itemset} se nessun immediato superinsieme ha esattamente lo stesso support count di X. In ogni caso il suo spporto deve essere $\ge minsup$.
\end{defn}
Importante qunado vi sono gruppi di prodotti venduti a blocco ignorando gli altri.

\textbf{Es}
\begin{figure}[H]
	\centering
	\includegraphics[height=0.5 \linewidth]{association/pict/close_freq_itemset.png}
	\caption{esempio di gruppi closed frequent itemset}
\end{figure}

Come si può notare in figura ogni gruppo di variabili (Gruppo A, B e C) è perfettamente associato e non hanno bisogno di mostrare items collegati ad altri gruppi. Assumendo che il $minsup = 20\%$, il numero totale di itemset frequenti è: $3 \cdot (2^5 -1) = 93$. In ogni caso vi sono solo 3 closed frequent itemset: 

$\{a1,a2,a3,a4,a5\}$

$\{b1,b2,b3,b4,b5\}$

$\{c1,c2,c3,c4,c5\}$


Questo tipo di itemset sono utili per rimuovere \textbf{regole di associazione ridondanti}. 
\begin{defn}
	Una regola di associazione $X \rightarrow Y$ \`e \textbf{ridondante} se esiste un'altra regola $X' \rightarrow Y'$ che rispetta certe propriet\`a. 
	\begin{itemize}
		\item $X \subseteq X'$
		\item $Y \subseteq Y'$
		\item $s(X \rightarrow Y)  = s(X' \rightarrow Y')$
		\item $c(X \rightarrow Y)  = c(X' \rightarrow Y')$
	\end{itemize}
\end{defn}

Mostriamo ora la gerarchia dei frequent itemset:
\begin{figure}[H]
	\centering
	\includegraphics[height=0.45 \linewidth]{association/pict/itemset_freq.png}
	\caption{gerarchia dei frequent itemset}
\end{figure}

Come si può notare i Maximal Frequent Itemset sono inclusi nei Closed Frequent Itemset perchè nessun maxmal può avere lo stesso support count del suo immediato superset.

\subsection{Rules Evaluation (*)}
Generati gli insiemi di pattern potenzialmente utili, bisogna ordinarli in base al loro livello di attrattività per il dominio di applicazione. La valutazione della qualità viene effettuata seconodo due criteri:
\begin{itemize}
	\item \textbf{Statistical arguments}: per patterns di items indipendenti e coperti da poche transazioni. Il problema è che \textit{possono catturare relazioni spurie} nei dati, per evitare:
	\begin{itemize}
		\item \underline{Objective Interestingness Measure}: usare statistiche derivate dai dati per determinare quali pattern sono interessanti
		\item \underline{Supporto, confidenza e correlazione}
	\end{itemize}
	\item \textbf{Subjective arguments}: un pattern è considerato non interessante a meno che riveli informazioni inaspettate riguardo ai dati e alla conoscenza: \underline{es}. forte associazione tra acquisto di pannolini e birra. È difficile fare questo tipo di valutazioni, richiede una considerevole quantità di informazioni pregresse dagli esperti di domino.
\end{itemize}

Meglio cercare di applicare un approccio più \textit{oggettivo} alla valutazione: data-driven, indipendente dal dominio in cui si richiede un minimo input dagli utenti (solo dei threshold) e calcolato basandosi sulle frequenze calcolate in una \textbf{Contingency Table}.
\begin{figure}[H]
	\centering
	\includegraphics[height=0.2 \linewidth]{association/pict/contingency_table.png}
	\caption{contingency table}
\end{figure}
dove:
\begin{itemize}
	\item $f_{1+}$ = support count di A
	\item $f_{+1}$ = support count di B
	\item $N$ = numero di transazioni totali
\end{itemize}

Supponiamo che un direttore di un minimarket voglia analizzare la relazione tra le persone che bevono tè e persone che bevono caffè. La seguente contingency table è ottenuta considerando le transazioni disponibili:
\begin{figure}[H]
	\centering
	\includegraphics[height=0.2 \linewidth]{association/pict/contingency_table_es.png}
	\caption{contingency table}
\end{figure}

Ora valutiamo l'associazione: tea $\rightarrow$ coffee
\begin{itemize}
	\item $support = \frac{f_{11}}{N} = \frac{150}{1000} = 15\%$
	\item $confidence = \frac{f_{11}}{f_{1+}} = \frac{150}{200} = 75\%$
\end{itemize}

Ad una prima occhiata sembrerebbe che le persone che bevono tè tendono a bere anche caffè (vedi confidenza). Ma se notiamo le persone che bevono caffè, a prescindere dal tè sono l'$80\%$ ($\frac{800}{1000}$), mentre la frazione dei bevitori di tè che bevono caffè è solo il $75\%$.

Da questo ragionamento si può concludere che il fatto di bere tè non influisca sulle persone che bevono caffè. Infatti nonostante l`associazione abbia un alto livello di confidenza ($75\%$) non si può ignorare il supporto dell'itemset conseguente ($80\%$).

Pertanto, vengono definiti altri indici:
\begin{defn}
	\textbf{Lift}: tasso di confidenza rispetto al supporto del conseguente
	
	\[ Lift = \frac{c(A \rightarrow B)}{s(B)}\]
\end{defn}

\begin{defn}
	\textbf{Interest Factor}: equivalente al \textit{Lift} ma per attributi binari, è definito in questo modo:
	
	\[ I(A,B)  = \frac{s(A,B)}{s(A)s(B)} = \frac{Nf_{11}}{f_{1+}f_{+1}}\]
\end{defn}
I valori sono così classificati:
\begin{itemize}
	\item $=1$ se A e B sono indipendenti
	\item $>1$ se A e B sono positivamente associati
	\item $<1$ se A e B sono negativamente associati
\end{itemize}
	
\begin{defn}
	\textbf{Analisi di correlazione} (per attributi binari simmetrici): analizza la relazione tra coppie di attributi. Attributi continui possono essere analizzati con la correlazione di Pearson, la correlazione per attributi binari è misurata usando il $\phi$-coefficient:

	\[\phi = \frac{f_{11}f_{00} - f_{01}f_{10}}{\sqrt{f_{1+}f_{+1}f_{0+}f_{+0}}}\]
	
	il valore varia da $[-1,+1]$. Se gli attributi sono statisticamente indipendenti il valore è $0$.
\end{defn}

\begin{defn}
	\textbf{IS Measure} (per attributi binari asimmetrici)
	
	\[IS(A,B) = \sqrt{I(A,B)s(A,B)} = \frac{s(A,B)}{\sqrt{s(A)s(B)}}\]
	
	Il suo valore è grande quanto l'\textit{Interest Factor} e il supporto sono grandi.
	
	Se A e B sono indipendenti:
	
	\[IS(A,B) = \sqrt{s(A)s(B)}\]
	
	ha lo stesso problema della correlazione, il valore può essere grande anche per associazioni incorrelate o negativamente correlate.
\end{defn}

Vi sono altri tipi di misure per l'analisi delle relazioni tra coppie di variabili binarie.

\begin{defn} \textbf{Misure Simmetriche}

Una misura \textbf{M} è \textbf{Simmetrica} se: $M(A \rightarrow B) = M(B \rightarrow A)$

es. Interest Factor è simmetrico
\end{defn}

\begin{defn}\textbf{Misure Asimmetriche}
	
Una misura \textbf{M} è \textbf{Asimmetrica} se: $M(A \rightarrow B) \ne M(B \rightarrow A)$

es. la Confidenza è asimmetrica
\end{defn}

Di seguito gli indici più utilizzati:

\begin{table}[H]
	\centering
	\begin{tabular}{|p{5cm}|p{5cm}|}
		\hline
		Simmetrico & Asimmetrico \\
		\hline
		Correlazione ($\phi$) & Gini index \\
		Odds ratio & Mutual information \\
		Kappa & Certainty factor \\
		Interest (I) & Added value \\
		Cosine (IS) & J-measure \\
		Jaccard & Goodman-Kruskal \\
		Collective strength & \\
		\hline
	\end{tabular}
\end{table}

È importante capire che ciascuna misura è adatta per analizzare un certo tipo di associazioni come basket market analysis o document analysis. In base al caso bisogna utilizzare gli indici migliori.\\

\textbf{NB}: sono stati presentati indici relativi a valutazioni per coppie di attributi binari, ma è possibile estendere l'analisi a più di due attributi usando tabelle delle frequenti in una Contingency Table multi-dimensionale. Gli indici come il Supporto Interest Factor e IS si prestano a ciò.

\subsection{Simpson's Paradox}

Consideriamo la seguente situazione:

un direttore di un mini-market racconta di una curiosa scoperta, legata agli item birra e hot dogs. Una società di consulenza da lui pagata per analizzare i suoi dati di vendita ha scoperto che i clienti che comprano birra sono meno tentati di acquistare hot dogs rispetto a quelli che non acquistano la birra. Il direttore però è convinto del contrario, lo sa per esperienza professionale che chi acquista birra tende ad acquistare hot dogs.

Come si può risolvere il paradosso?

Consideriamo la seguente tabella:
\begin{figure}[H]
	\centering
	\includegraphics[height=0.2 \linewidth]{association/pict/beer_hotdog.png}
	\caption{es. birra - hot dog}
\end{figure}
Consideriamo le seguenti regole con le relative confidenze:
\begin{itemize}
	\item \{beer\} $\rightarrow$ \{hot dogs\} - confidence = $42\%$
	\item \{NOT beer\} $\rightarrow$ \{hot dogs\} - confidence = $45\%$
\end{itemize}	
Possiamo inferire che i clienti che acquistano birra sono meno inclini ($42\%$) ad acquistare hot dog rispetto a quelli che non acquistano birra ($45\%$).

Analizziamo ora gli stessi dati ma categorizzando se il cliente è single o no:
\begin{figure}[H]
	\centering
	\includegraphics[height=0.2 \linewidth]{association/pict/beer_hotdog_single.png}
	\caption{es. cliente single: birra - hot dog}
\end{figure}
Pertanto posso derivare queste due coppie di regole:
\begin{itemize}
	\item Single
	\begin{itemize}
		\item \{beer\} $\rightarrow$ \{hot dogs\} - confidence = $75\%$
		\item \{NOT beer\} $\rightarrow$ \{hot dogs\} - confidence = $11\%$
	\end{itemize}
	\item NOT Single
	\begin{itemize}
		\item \{beer\} $\rightarrow$ \{hot dogs\} - confidence = $31\%$
		\item \{NOT beer\} $\rightarrow$ \{hot dogs\} - confidence = $73\%$
	\end{itemize}
\end{itemize}
Da questi dati posso affermare che: i single che acqustano birra sono più inclini ($75\%$) ad acquistare anche hot dog rispetto a quelli che non acquistano birra ($11\%$).\\

Quindi sia l'azienda di consulenza che il direttore del mini-market avevano ragione soltanto che per comprenderlo bisognava categorizzare i clienti. 

\paragraph{Paradosso di Simpson} o Yule-Simpson effect, è un paradosso della probabilità e statistica, in cui una tendenza appare in diversi gruppi di dati ma sparisce o si inverte quando questi gruppi sono combinati. \\

Bisogna applicare una appropriata stratificazione dei dati per evitare la generazione di associazioni spurie risultati dal paradosso di Simpson. 

Es. per i dati del market-basket, la catena di supermercati dovrebbe stratificarli secondo la location del negozio, mentre i dati sanitari da vari pazienti dovrebbero essere stratificati secondo fattori quali l'età o il genere.






\end{document}
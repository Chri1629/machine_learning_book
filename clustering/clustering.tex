\restylefloat{table,figure}
\pagestyle{fancy}
\cfoot{\thepage}
\renewcommand{\footrulewidth}{0.25pt}


\section{Introduzione}
Gli ambiti più importanti nei quali vengono applicate tecniche di machine learning nella vita reale sono:
\begin{itemize}
	\item Finanza
	\item Sanità
	\item Agricoltura
	\item e-commerce
	\item Social
	\item Chatbot
	\item Sensoristica (come i veicoli a guida autonoma)
\end{itemize}

Vista la grande mole di dati la necessità è capire come trattare i dati.

\textbf{L'obiettivo del Machine Learning è sviluppare metodologia per dare valore ai dati in funzione di una particolare domanda che ci stiamo facendo.}

Tipicamente le tecninche di machine learning si dividono nelle seguenti tre macro categorie:
\begin{itemize}
	\item Apprendimento supervisionato o predittivo: qualcuno ha gia catalogato ad esempio delle immagini o dei dati e noi prendendo questi modelli dovremmo essere in grado di predire.
	\item Apprendimento descrittivo: ci sono delle funzioni obiettivo che vanno ottimizzate. Non usiamo etichette della singola istanza ma in qualche modo sappiamo dove arrivare.
	\item Apprendimento rinforzato: E' quello più utile in questa epoca: funziona sui premi.
\end{itemize}

In questo corso ci concentreremo sui primi due tipi di apprendimento.

L' \textit{apprendimento supervisionato} si divide a sua volta nelle seguenti due categorie:
\begin{itemize}
	\item Classificazione: quando il problema consiste nel dividere in classi delle quantità discrete.
	\item Regressione: quando il problema consiste nel ricostruire una certa variabile date delle condizioni pregresse.
\end{itemize}

L' \textit{apprendimento non supervisionato} si divide a sua volta in:
\begin{itemize}
	\item Clustering: Quando il problema consistere nel ricostruire delle classi delle istanze che ci vengono consegnate, senza sapere nulla sulla storia pregressa.
	\item Associativa: Quando il problema consiste nel scoprire pattern che descrivono bene caratteristiche associate ad un certo fenomeno.
\end{itemize}

Per alcuni compiti la correlazione statistica va benissimo, in alcuni casi però essa è addirittura deleteria.  Se infatti provassi a vedere la correlazione tra il numero di omicidi in america e il numero di fondi investiti sulla ricerca scientifica vedrei che statisticamente sono strettamente correlate. Questo è un no-sense ed è il classico esempio di \textit{correlazione spuria.}

A confermare questa visione in cui il modello per una certa trattazione dati è fondamentale è il cosidetto paradosso di Simpson.

\textbf{Paradosso di Simpson:} Se uso solo i dati senza modello no c'è alcun modo di scoprire la verità.


\subsection{Data Types}

Il primo passo fondamentale è sicuramente quello di prendere confidenza coi dati. E' fondamentale capire la natura dei dati che abbiamo a disposizione (\textbf{dataset}), c'è un fenomeno che si chiama \textbf{churn}, quando non siamo soddisfatti di un servizio ci affidiamo al competitor.

Ci possono essere valori missing in un dataset e possono mancare per diversi motivi.

Le colonne sono chiamate \textbf{Attributi.}

Le righe sono chiamate \textbf{Istanze.}

A volte ci sono anche attributi duplicati che possiamo tranquillamente buttare fuori.
Ogni attributo è caratterizzato dal fatto di avere un tipo. Conoscerlo è fondamentale per trattare i dati.
Gli attributi si dividono in due grandi gruppi:
\begin{itemize}
	\item Categorici:
	\begin{itemize}
		\item Nominali: ad esempio il colore degli occhi.
		\item Ordinali: ad esempio possono essere i giudizi.
	\end{itemize}
	\item Numerici:
	\begin{itemize}
		\item Intervallo: ammettono operazioni di somma e sottrazione. 
		\item Ratio: possiamo applicare tutte le operazioni logico/matematiche.
	\end{itemize}
\end{itemize}

Dall'alto al basso il livello gerarchico sale e le proprietà aumentano.

Si possono anche dividere in attributi \textit{discreti} che possono essere:
\begin{itemize}
	\item Categorici
	\item Numerici
	\item Binari: sono i più particolari da trattare, e hanno una serie di proprietà strane.
\end{itemize}
Oppure possono essere \textit{Continui.}

\subsection{Data exploration}

Dobbiamo però anche sapere come esplorare i dati. Per farlo facciamo cose molto elementari.
Per farlo si usano tutti gli strumenti a nostra disposizione.

Il concetto di \textbf{quantile} è trovare il numero di osservazione che ci indica quanti attributi sono più piccoli di un dato valore.
Un quantile molto importante è il quantile di ordine $\frac{1}{2}$ e si chiama \textbf{Mediana} che è quello che ha esattamente minori di lui la metà dei dati.

\[mean = \frac{1}{n}\sum_{i=1}^{n}x_i\]

La media non è un buon modo di visualizzare i dati perché dice poco ma quanto meno dice qualcosa. Siccome la media è basata su singole osservazioni si possono vedere le presenze di outliar, ossia di elementi troppo discordanti dalla media e che si presenta poche volte. Sono quindi elementi che è necessario trattare per vedere la provenienza.

Per prevenire questo si usa la \textbf{media trimmed} in cui si buttano via il valore più piccolo e il valore più grande. Se si trova un grosso scostamento probabilmente è presente un outlier.

Si puù definire anche il range anche se di solito si usa la varianza:
\[var = \frac{1}{n}\sum_{i = 1}^{n} (x_{i} - \bar{x})^{2}\]

Di solito si usa la deviazione standard che ha lo stesso ordine di grandezza dei dati:

\[ std = \sqrt{var} \]

Si usa anche il range interquartile (IQR) sempre per ovviare alla presenza di outliar.
Se ho a che fare con coppie di attributi allora è naturale parlare di \textbf{covarianza} ossia la varianza calcolata su due attributi diversi:

\[cov(X,Y) = \frac{1}{n}\sum_{i = 1}^{n} (x_{i} - \bar{x})(y_{i} - \bar{y})\]

Di solito si fanno scalature perché sennò le covarianze vengono troppo sbagliate. Per ovviare uso la \textbf{correlazione di Pearson} che può prendere valori $[-1,1]$
\[ corr(x,y) = \frac{cov(x,y)}{\sqrt{var(x)var(y)}}\]

Possono organizzare i dati in istogrammi in cui posso usare una ampiezza fissa o variabile (bin). A seconda dell'ampiezza che uso posso ottenere due disegni molto diversi.

Un altro modo di rappresentare i dati particolarmente utile è il \textbf{grafico Box and Whiskers} applicato solo ad attributi quantitativi.

\subsection{Missing replacement}

E' un problema enorme di per sè. Le operazioni più elementari sono le seguenti. In alcuni valori degli attributi un valore non è registrato. Ci possono esser tante ragioni: ad esempio un attributo non è sempre stato osservabile (penso all'ambito clinico), o ad esempio un attributo prima non veniva considerato rilevante.

Il primo nmetodo è il \textbf{Record removal} che è molto drastico come metodo perché comunque sia vengono eliminati dei valori che sarebbero potuti essere molto importanti.

Il secondo metodo è quello di \textbf{imputazione manuale}: è fatta da umani e tramite osservazioni ci si chiede se sia possibile inserirlo, è tremendamente difficile dal punto di vista computazionale.

Il terzo metodo è quello della \textbf{global constant}: ossia metto un numero là chiamato place holder con un valore costante, non troppo efficiente.

Il quarto metodo è quello di \textbf{rimpiazzarlo con la moda}, anche questo però è fortemente criticabile.
Se gli attributi sono continui si fa la stessa cosa ma con la media.

Il quinto metodo è \textbf{Conditional mean replacement} ossia bisogna rimpiazzare con la media solo se è presente un altro determinato attributo.

Il sesto metodo è quello del \textbf{most probable} ossia di prendere un modello e sostituire il valore.

E' molto difficile dare il confine tra l'esplorazione dei dati e la modellizzazione dei dati.

\subsection{Data Preprocessing}

Tutti questi processi impattano fortemente tutta l'analisi che farò dopo.



